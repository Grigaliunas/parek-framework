% Options for packages loaded elsewhere
\PassOptionsToPackage{unicode}{hyperref}
\PassOptionsToPackage{hyphens}{url}
\documentclass[
  english,
]{article}
\usepackage{xcolor}
\usepackage{amsmath,amssymb}
\setcounter{secnumdepth}{5}
\usepackage{iftex}
\ifPDFTeX
  \usepackage[T1]{fontenc}
  \usepackage[utf8]{inputenc}
  \usepackage{textcomp} % provide euro and other symbols
\else % if luatex or xetex
  \usepackage{unicode-math} % this also loads fontspec
  \defaultfontfeatures{Scale=MatchLowercase}
  \defaultfontfeatures[\rmfamily]{Ligatures=TeX,Scale=1}
\fi
\usepackage{lmodern}
\ifPDFTeX\else
  % xetex/luatex font selection
\fi
% Use upquote if available, for straight quotes in verbatim environments
\IfFileExists{upquote.sty}{\usepackage{upquote}}{}
\IfFileExists{microtype.sty}{% use microtype if available
  \usepackage[]{microtype}
  \UseMicrotypeSet[protrusion]{basicmath} % disable protrusion for tt fonts
}{}
\makeatletter
\@ifundefined{KOMAClassName}{% if non-KOMA class
  \IfFileExists{parskip.sty}{%
    \usepackage{parskip}
  }{% else
    \setlength{\parindent}{0pt}
    \setlength{\parskip}{6pt plus 2pt minus 1pt}}
}{% if KOMA class
  \KOMAoptions{parskip=half}}
\makeatother
\usepackage{color}
\usepackage{fancyvrb}
\newcommand{\VerbBar}{|}
\newcommand{\VERB}{\Verb[commandchars=\\\{\}]}
\DefineVerbatimEnvironment{Highlighting}{Verbatim}{commandchars=\\\{\}}
% Add ',fontsize=\small' for more characters per line
\newenvironment{Shaded}{}{}
\newcommand{\AlertTok}[1]{\textcolor[rgb]{1.00,0.00,0.00}{\textbf{#1}}}
\newcommand{\AnnotationTok}[1]{\textcolor[rgb]{0.38,0.63,0.69}{\textbf{\textit{#1}}}}
\newcommand{\AttributeTok}[1]{\textcolor[rgb]{0.49,0.56,0.16}{#1}}
\newcommand{\BaseNTok}[1]{\textcolor[rgb]{0.25,0.63,0.44}{#1}}
\newcommand{\BuiltInTok}[1]{\textcolor[rgb]{0.00,0.50,0.00}{#1}}
\newcommand{\CharTok}[1]{\textcolor[rgb]{0.25,0.44,0.63}{#1}}
\newcommand{\CommentTok}[1]{\textcolor[rgb]{0.38,0.63,0.69}{\textit{#1}}}
\newcommand{\CommentVarTok}[1]{\textcolor[rgb]{0.38,0.63,0.69}{\textbf{\textit{#1}}}}
\newcommand{\ConstantTok}[1]{\textcolor[rgb]{0.53,0.00,0.00}{#1}}
\newcommand{\ControlFlowTok}[1]{\textcolor[rgb]{0.00,0.44,0.13}{\textbf{#1}}}
\newcommand{\DataTypeTok}[1]{\textcolor[rgb]{0.56,0.13,0.00}{#1}}
\newcommand{\DecValTok}[1]{\textcolor[rgb]{0.25,0.63,0.44}{#1}}
\newcommand{\DocumentationTok}[1]{\textcolor[rgb]{0.73,0.13,0.13}{\textit{#1}}}
\newcommand{\ErrorTok}[1]{\textcolor[rgb]{1.00,0.00,0.00}{\textbf{#1}}}
\newcommand{\ExtensionTok}[1]{#1}
\newcommand{\FloatTok}[1]{\textcolor[rgb]{0.25,0.63,0.44}{#1}}
\newcommand{\FunctionTok}[1]{\textcolor[rgb]{0.02,0.16,0.49}{#1}}
\newcommand{\ImportTok}[1]{\textcolor[rgb]{0.00,0.50,0.00}{\textbf{#1}}}
\newcommand{\InformationTok}[1]{\textcolor[rgb]{0.38,0.63,0.69}{\textbf{\textit{#1}}}}
\newcommand{\KeywordTok}[1]{\textcolor[rgb]{0.00,0.44,0.13}{\textbf{#1}}}
\newcommand{\NormalTok}[1]{#1}
\newcommand{\OperatorTok}[1]{\textcolor[rgb]{0.40,0.40,0.40}{#1}}
\newcommand{\OtherTok}[1]{\textcolor[rgb]{0.00,0.44,0.13}{#1}}
\newcommand{\PreprocessorTok}[1]{\textcolor[rgb]{0.74,0.48,0.00}{#1}}
\newcommand{\RegionMarkerTok}[1]{#1}
\newcommand{\SpecialCharTok}[1]{\textcolor[rgb]{0.25,0.44,0.63}{#1}}
\newcommand{\SpecialStringTok}[1]{\textcolor[rgb]{0.73,0.40,0.53}{#1}}
\newcommand{\StringTok}[1]{\textcolor[rgb]{0.25,0.44,0.63}{#1}}
\newcommand{\VariableTok}[1]{\textcolor[rgb]{0.10,0.09,0.49}{#1}}
\newcommand{\VerbatimStringTok}[1]{\textcolor[rgb]{0.25,0.44,0.63}{#1}}
\newcommand{\WarningTok}[1]{\textcolor[rgb]{0.38,0.63,0.69}{\textbf{\textit{#1}}}}
\usepackage{longtable,booktabs,array}
\usepackage{calc} % for calculating minipage widths
% Correct order of tables after \paragraph or \subparagraph
\usepackage{etoolbox}
\makeatletter
\patchcmd\longtable{\par}{\if@noskipsec\mbox{}\fi\par}{}{}
\makeatother
% Allow footnotes in longtable head/foot
\IfFileExists{footnotehyper.sty}{\usepackage{footnotehyper}}{\usepackage{footnote}}
\makesavenoteenv{longtable}
\ifLuaTeX
\usepackage[bidi=basic,shorthands=off,]{babel}
\else
\usepackage[bidi=default,shorthands=off,]{babel}
\fi
\ifLuaTeX
  \usepackage{selnolig} % disable illegal ligatures
\fi
\setlength{\emergencystretch}{3em} % prevent overfull lines
\providecommand{\tightlist}{%
  \setlength{\itemsep}{0pt}\setlength{\parskip}{0pt}}
% Final fallback header.tex — works with MacTeX + XeLaTeX on macOS

\usepackage{fontspec}
\setmainfont{Latin Modern Roman}
\setsansfont{Latin Modern Sans}
\setmonofont{Latin Modern Mono}

\usepackage{unicode-math}
\setmathfont{Latin Modern Math}

\usepackage{pifont}
\newcommand{\cmark}{\ding{51}}     % ✔
\newcommand{\xmark}{\ding{55}}     % ✘
\newcommand{\checkbox}{\ding{113}} % ☐
\newcommand{\radiobox}{\ding{109}} % ◉
\newcommand{\arrowmark}{\ding{228}} % →

\usepackage{amsmath, amssymb}
\usepackage{setspace}
\onehalfspacing

\usepackage{fancyhdr}
\pagestyle{fancy}
\fancyhf{}
\fancyhead[L]{PAREK Framework}
\fancyhead[R]{\today}
\fancyfoot[C]{\thepage}
\usepackage{bookmark}
\IfFileExists{xurl.sty}{\usepackage{xurl}}{} % add URL line breaks if available
\urlstyle{same}
\hypersetup{
  pdftitle={PAREK Framework -- EU Post‑Quantum Cryptography Transition Handbook},
  pdflang={en},
  hidelinks,
  pdfcreator={LaTeX via pandoc}}

\title{PAREK Framework -- EU Post‑Quantum Cryptography Transition
Handbook}
\author{}
\date{2025-07-02}

\begin{document}
\maketitle

{
\setcounter{tocdepth}{2}
\tableofcontents
}
\section{1 Document Control \& Revision
History}\label{document-control-revision-history}

\begin{quote}
This file provides the authoritative revision history for the
\emph{PAREK Framework -- EU Post‑Quantum Cryptography Transition
Handbook}. Update \textbf{only} via pull‑request. Each entry must be
approved by the Handbook Steering Committee.
\end{quote}

\begin{longtable}[]{@{}
  >{\raggedright\arraybackslash}p{(\linewidth - 8\tabcolsep) * \real{0.0648}}
  >{\raggedright\arraybackslash}p{(\linewidth - 8\tabcolsep) * \real{0.1481}}
  >{\raggedright\arraybackslash}p{(\linewidth - 8\tabcolsep) * \real{0.1944}}
  >{\raggedright\arraybackslash}p{(\linewidth - 8\tabcolsep) * \real{0.2685}}
  >{\raggedright\arraybackslash}p{(\linewidth - 8\tabcolsep) * \real{0.3241}}@{}}
\toprule\noalign{}
\begin{minipage}[b]{\linewidth}\raggedright
Version
\end{minipage} & \begin{minipage}[b]{\linewidth}\raggedright
Date (YYYY‑MM‑DD)
\end{minipage} & \begin{minipage}[b]{\linewidth}\raggedright
Editor
\end{minipage} & \begin{minipage}[b]{\linewidth}\raggedright
Section(s) changed
\end{minipage} & \begin{minipage}[b]{\linewidth}\raggedright
Change description
\end{minipage} \\
\midrule\noalign{}
\endhead
\bottomrule\noalign{}
\endlastfoot
0.1 & 2025‑06‑24 & PAREK Editorial Team & Initial skeleton & Created
document control template \\
\end{longtable}

\subsubsection{How to update}\label{how-to-update}

\begin{enumerate}
\def\labelenumi{\arabic{enumi}.}
\tightlist
\item
  Increment the \textbf{version} number using semantic format (e.g.,
  \texttt{0.2}, \texttt{1.0}).\\
\item
  Add a concise \textbf{change description} (≤~100~characters).\\
\item
  If multiple sections change, list comma‑separated values in
  \textbf{Section(s) changed)}.\\
\item
  Commit the file and open a pull request tagged
  \texttt{\#document‑control}.
\end{enumerate}

\section{2 Executive Summary}\label{executive-summary}

\begin{quote}
\textbf{Purpose of this section} -- provide senior stakeholders with a
concise, non‑technical overview of the quantum‑threat context, the PAREK
Framework's objectives, and the high‑level roadmap that underpins the
handbook. This one‑pager should be intelligible to board members,
regulators and project sponsors.
\end{quote}

\subsection{Executive Summary (draft placeholder -- to be expanded in
v0.2)}\label{executive-summary-draft-placeholder-to-be-expanded-in-v0.2}

The advent of \textbf{cryptographically‑relevant quantum computers
(CRQC)} will render today's RSA and ECC protections ineffective, putting
long‑lived confidential data at risk of \emph{harvest‑now‑decrypt‑later}
attacks. To safeguard digital sovereignty and comply with forthcoming EU
mandates, organisations must transition to \textbf{post‑quantum
cryptography (PQC)} well before widely available CRQC capabilities
emerge.

The \textbf{PAREK Framework} offers a five‑stage
lifecycle---\textbf{P}ost‑quantum asset \& algorithm inventory,
\textbf{A}ssessment of quantum risk, \textbf{R}oad‑mapping \& readiness
planning, \textbf{E}xecution \& migration, and \textbf{K}ey‑governance
\& continuous improvement. It aligns with the EU Coordinated
Implementation Roadmap timelines (inventory baseline by 2026, high‑risk
cut‑over by 2030, and medium‑risk completion by 2035) and integrates
NIST FIPS‑validated algorithms (ML‑KEM, ML‑DSA, SPHINCS+).

Successful adoption hinges on three pillars: 1. \textbf{Comprehensive
discovery} of cryptographic assets (CBOMs) across the entire estate. 2.
\textbf{Risk‑based prioritisation} using the Quantum‑Adjusted Risk Score
(QARS). 3. \textbf{Supplier integration} via contract clauses and
machine‑readable attestations.

By following PAREK, the organisation will achieve crypto‑agility,
maintain regulatory compliance, and preserve stakeholder trust in the
quantum era.

\emph{(This text is a high‑level placeholder. Subsequent revisions will
incorporate quantitative risk metrics, budget highlights and KPI
snapshots once sections 8--15 are finalised.)}

\section{3 Purpose, Scope \& Audience}\label{purpose-scope-audience}

\begin{quote}
\textbf{Purpose of this section} -- clarify \emph{why} the handbook
exists, \emph{what} systems and data it covers, and \emph{who} should
read and apply its guidance.
\end{quote}

\subsection{3.1 Purpose (placeholder)}\label{purpose-placeholder}

The \emph{PAREK Framework Handbook} establishes a common, evidence‑based
approach for migrating the organisation's cryptography to post‑quantum
algorithms in alignment with EU regulatory timelines and industry best
practice. It consolidates policies, processes and technical playbooks
into a single authoritative source.

\subsection{3.2 Scope (placeholder)}\label{scope-placeholder}

\begin{quote}
\emph{To be detailed in v0.2 once the asset inventory (§8) and risk
classification (§9) are baselined.}
\end{quote}

\begin{longtable}[]{@{}
  >{\raggedright\arraybackslash}p{(\linewidth - 4\tabcolsep) * \real{0.2308}}
  >{\raggedright\arraybackslash}p{(\linewidth - 4\tabcolsep) * \real{0.1385}}
  >{\raggedright\arraybackslash}p{(\linewidth - 4\tabcolsep) * \real{0.6308}}@{}}
\toprule\noalign{}
\begin{minipage}[b]{\linewidth}\raggedright
Domain
\end{minipage} & \begin{minipage}[b]{\linewidth}\raggedright
In scope?
\end{minipage} & \begin{minipage}[b]{\linewidth}\raggedright
Notes (draft)
\end{minipage} \\
\midrule\noalign{}
\endhead
\bottomrule\noalign{}
\endlastfoot
Production apps & ✔ & All customer‑facing and back‑office apps \\
Dev/test envs & ✔ & CI/CD pipelines, test data masks \\
OT/ICS networks & ☐ & Pending risk assessment outcome \\
Third‑party SaaS & ✔ & Subject to CBOM/contract clauses (see §13) \\
Legacy mainframe & ☐ & Migration feasibility under investigation \\
\end{longtable}

\subsection{3.3 Audience (placeholder)}\label{audience-placeholder}

Primary readers: - \textbf{Executive sponsors} (CIO, CISO, CRO) --
governance \& budget - \textbf{Security architects / cryptographers} --
technical standards - \textbf{Product \& dev‑ops teams} --
implementation guidance - \textbf{Procurement \& legal} -- supplier
clauses, contract annexes - \textbf{Regulators \& auditors} --
compliance evidence

Secondary readers: - Vendors, open‑source maintainers, academic
reviewers.

\begin{quote}
Subsequent revisions will flesh out scope exclusions, detailed audience
personas, and cross‑references to internal policies once sections 4--15
mature.
\end{quote}

\section{4 Regulatory \& Strategic
Context}\label{regulatory-strategic-context}

\begin{quote}
\textbf{Purpose of this section} -- highlight the EU directives,
national regulations and international standards that drive the
organisation's post‑quantum transition, and explain how the PAREK
Framework aligns with these external obligations.
\end{quote}

\subsection{\texorpdfstring{4.1 EU regulatory landscape
\emph{(placeholder)}}{4.1 EU regulatory landscape (placeholder)}}\label{eu-regulatory-landscape-placeholder}

\begin{itemize}
\tightlist
\item
  \textbf{NIS~2 Directive} -- security and reporting duties for
  essential/important entities.\\
\item
  \textbf{Cyber Resilience Act (CRA)} -- forthcoming product‑security
  requirements incl.~cryptographic transparency.\\
\item
  \textbf{EU Coordinated Implementation Roadmap for PQC (2025)} -- joint
  milestones: inventory 2026, high‑risk cut‑over 2030, medium‑risk 2035.
\end{itemize}

\emph{(Detailed mapping to be added when national transposition
timelines are confirmed.)}

\subsection{\texorpdfstring{4.2 Strategic alignment
\emph{(placeholder)}}{4.2 Strategic alignment (placeholder)}}\label{strategic-alignment-placeholder}

Describe how the organisation's cyber‑security strategy,
data‑classification policy and digital‑sovereignty goals intersect with
PQC adoption.

\subsection{\texorpdfstring{4.3 External standards map
\emph{(placeholder)}}{4.3 External standards map (placeholder)}}\label{external-standards-map-placeholder}

\begin{longtable}[]{@{}
  >{\raggedright\arraybackslash}p{(\linewidth - 4\tabcolsep) * \real{0.4318}}
  >{\raggedright\arraybackslash}p{(\linewidth - 4\tabcolsep) * \real{0.1591}}
  >{\raggedright\arraybackslash}p{(\linewidth - 4\tabcolsep) * \real{0.4091}}@{}}
\toprule\noalign{}
\begin{minipage}[b]{\linewidth}\raggedright
Standard
\end{minipage} & \begin{minipage}[b]{\linewidth}\raggedright
Status
\end{minipage} & \begin{minipage}[b]{\linewidth}\raggedright
Relevance to PAREK
\end{minipage} \\
\midrule\noalign{}
\endhead
\bottomrule\noalign{}
\endlastfoot
\textbf{NIST FIPS~203‑205} & Final & Baseline algorithms (ML‑KEM,
ML‑DSA, SPHINCS+) \\
ISO/IEC~DIS~14888‑4 & Draft & PQ signatures \\
ETSI~TS~103~829 & Stable & Hybrid key exchange \\
\end{longtable}

\emph{(Will expand once draft texts are ratified.)}

\begin{quote}
Subsequent revisions will add jurisdiction‑specific compliance
deadlines, cross‑reference to §15 KPIs, and commentary on industry
guidance (e.g., ENISA reports).
\end{quote}

\section{5 Quantum Threat Landscape}\label{quantum-threat-landscape}

\begin{quote}
\textbf{Purpose of this chapter} -- present an evidence‑based assessment
of how, when and why quantum computing threatens today's cryptographic
defences, and establish the urgency that underpins every subsequent
stage of the PAREK Framework.
\end{quote}

\subsection{5.1 Executive overview}\label{executive-overview}

A new generation of \textbf{cryptographically‑relevant quantum computers
(CRQC)} threatens to break RSA and elliptic‑curve public‑key
cryptography, as well as reduce the effective security of some symmetric
systems. Although no public demonstration of large‑scale key‑recovery
exists as of \emph{June~2025}, the physics, engineering and economic
trends analysed in this chapter indicate that organisations must
complete the transition to post‑quantum cryptography (PQC) \textbf{well
before 2035} to avert the twin risks of \emph{harvest‑now‑decrypt‑later}
(HNDL) attacks and regulatory sanction.

Key messages:

\begin{itemize}
\tightlist
\item
  Commercial hardware roadmaps (IBM ``Kookaburra'' 1,386‑qubit chip,
  planned for late~2025) illustrate a \textbf{quadratic growth curve}
  comparable to early classical Moore's
  Law~(\href{https://www.ibm.com/quantum/blog/ibm-quantum-roadmap-2025?utm_source=chatgpt.com}{ibm.com}).
\item
  Expert‑elicitation studies (Global Risk Institute \emph{Quantum Threat
  Timeline 2024 \&~2025}) put the median arrival of a CRQC capable of
  breaking RSA‑2048 in the \textbf{early‑to‑mid 2030s}, with a 10\,\%
  probability the event occurs \textbf{before
  2030}~(\href{https://globalriskinstitute.org/publication/2024-quantum-threat-timeline-report/?utm_source=chatgpt.com}{globalriskinstitute.org},
  \href{https://globalriskinstitute.org/publication/quantum-threat-timeline-2025-executive-perspectives-on-barriers-to-action/?utm_source=chatgpt.com}{globalriskinstitute.org}).
\item
  Real‑world HNDL behaviour is now documented across sectors such as
  maritime logistics and financial
  services~(\href{https://www.marinelink.com/news/harvest-decrypt-later-526089?utm_source=chatgpt.com}{marinelink.com},
  \href{https://www.keyfactor.com/blog/harvest-now-decrypt-later-a-new-form-of-attack/?utm_source=chatgpt.com}{keyfactor.com}).
\item
  Regulators have moved from guidance to \textbf{mandatory timelines}
  (e.g., US OMB M‑23‑02, EU Coordinated Implementation Roadmap, Europol
  Quantum Safe Financial
  Forum)~(\href{https://www.reuters.com/technology/cybersecurity/europol-body-banks-should-prepare-quantum-computer-risk-now-2025-02-07/?utm_source=chatgpt.com}{reuters.com}).
\end{itemize}

The remainder of this chapter unpacks these trends and quantifies the
residual uncertainty.

\begin{center}\rule{0.5\linewidth}{0.5pt}\end{center}

\subsection{5.2 From laboratory curiosity to
CRQC}\label{from-laboratory-curiosity-to-crqc}

A CRQC is not just a bigger quantum processor; it must combine
\textbf{millions of physical qubits}, fast classical co‑processing and
robust error correction to implement Shor's algorithm at scale. The
consensus path involves:

\begin{enumerate}
\def\labelenumi{\arabic{enumi}.}
\tightlist
\item
  \textbf{Hardware scaling} -- IBM's 1,121‑qubit \emph{Condor} (2024)
  and planned 4,158‑qubit multi‑chip Kookaburra system
  (2025‑26)~(\href{https://www.ibm.com/quantum/blog/quantum-roadmap-2033?utm_source=chatgpt.com}{ibm.com},
  \href{https://www.ibm.com/quantum/blog/ibm-quantum-roadmap-2025?utm_source=chatgpt.com}{ibm.com}).
\item
  \textbf{Error‑correction breakthroughs} -- low‑overhead surface codes
  + lattice surgery lowering logical‑to‑physical ratios by 30‑50\,\%
  (published Nature, Feb~2025).
\item
  \textbf{Interconnects \& parallelism} -- photonic links to cluster
  cryostats, already demonstrated in AWS Braket prototypes.
\end{enumerate}

Resource‑estimation papers (Gidney \&~Ekerå 2023) suggest breaking
RSA‑2048 would require \textasciitilde20\,M physical qubits running for
8\,hours at 10−3 physical error rates. The delta between current
prototypes and this target is shrinking annually by \textbf{1‑2 orders
of magnitude}.

\subsection{5.3 Threat timeline
projections}\label{threat-timeline-projections}

\subsubsection{5.3.1 Survey‑based
forecasts}\label{surveybased-forecasts}

The Global Risk Institute's \emph{2024 Quantum Threat Timeline} surveyed
61 experts across academia and industry. Results (Figure~1) assign:

\begin{itemize}
\tightlist
\item
  10\,\% probability of CRQC by \textbf{2029}
\item
  50\,\% probability by \textbf{2033‑2035}
\item
  90\,\% probability by \textbf{2039‑2040}
\end{itemize}

An updated \emph{2025 Executive Perspective} report, focusing on
financial‑sector CISOs, reveals that \textbf{one‑third of respondents
shortened their internal ``must‑migrate‑by'' date by 2~years} compared
with
2023~(\href{https://globalriskinstitute.org/publication/quantum-threat-timeline-2025-executive-perspectives-on-barriers-to-action/?utm_source=chatgpt.com}{globalriskinstitute.org}).

\subsubsection{5.3.2 Engineering trend
extrapolation}\label{engineering-trend-extrapolation}

IBM's roadmap shows qubit count doubling roughly every 18~months since
2017. If sustained, a 2‑M qubit device (roughly RSA‑2048 breaking
threshold) is plausible by \textbf{2031‑2033}. While \emph{hardware
alone is not destiny}, software stack and cryogenics must co‑evolve; yet
venture‑capital funding ballooned to USD~4.2\,B in 2024, signalling
market capacity to close those gaps.

\subsection{5.4 Harvest‑now‑decrypt‑later
evidence}\label{harvestnowdecryptlater-evidence}

Analysts at Keyfactor and Mandiant observe APT groups stockpiling
TLS‑encrypted session captures and VPN archives since at least 2021.
Shipping‑sector telemetry from Marlink (Q4~2024) logged nine billion
encrypted packets exfiltrated and stored in off‑net
buckets~(\href{https://www.keyfactor.com/blog/harvest-now-decrypt-later-a-new-form-of-attack/?utm_source=chatgpt.com}{keyfactor.com},
\href{https://www.marinelink.com/news/harvest-decrypt-later-526089?utm_source=chatgpt.com}{marinelink.com}).
Although current classical resources cannot decrypt them, \textbf{data
confidentiality lifetimes}---especially in finance, healthcare and
national security---often exceed 25~years, bridging the gap to plausible
CRQC dates.

\subsection{5.5 Regulatory accelerants}\label{regulatory-accelerants}

\begin{longtable}[]{@{}
  >{\raggedright\arraybackslash}p{(\linewidth - 4\tabcolsep) * \real{0.4444}}
  >{\raggedright\arraybackslash}p{(\linewidth - 4\tabcolsep) * \real{0.2593}}
  >{\raggedright\arraybackslash}p{(\linewidth - 4\tabcolsep) * \real{0.2963}}@{}}
\toprule\noalign{}
\begin{minipage}[b]{\linewidth}\raggedright
Jurisdiction
\end{minipage} & \begin{minipage}[b]{\linewidth}\raggedright
Mandate
\end{minipage} & \begin{minipage}[b]{\linewidth}\raggedright
Deadline
\end{minipage} \\
\midrule\noalign{}
\endhead
\bottomrule\noalign{}
\endlastfoot
\textbf{United States} & OMB M‑23‑02: agencies submit PQC inventory →
migrate high‑impact systems & Inventory 2027; migration end‑2035 \\
\textbf{European Union} & Coordinated Roadmap: inventory baseline,
high‑risk cut‑over & 2026; 2030; 2035 \\
\textbf{Brazil} & Central Bank circular on quantum‑safe data storage &
2032 \\
\textbf{Global finance} & Europol‑backed Quantum Safe Financial Forum
urges ``prepare now'' & Guidance
Feb~2025~(\href{https://www.reuters.com/technology/cybersecurity/europol-body-banks-should-prepare-quantum-computer-risk-now-2025-02-07/?utm_source=chatgpt.com}{reuters.com}) \\
\end{longtable}

NIST cemented the algorithm baseline with \textbf{FIPS 203 (ML‑KEM), 204
(ML‑DSA) and 205 (SPHINCS+)} in August\,2024, removing a key blocker to
production
rollout~(\href{https://csrc.nist.gov/news/2024/postquantum-cryptography-fips-approved?utm_source=chatgpt.com}{csrc.nist.gov}).

\subsection{5.6 Sector‑specific impact
analysis}\label{sectorspecific-impact-analysis}

\subsubsection{5.6.1 Finance}\label{finance}

\begin{itemize}
\tightlist
\item
  Long data retention (KYC, trade archives) + high Target~Value (TV) ⇒
  migration priority.
\item
  Real‑time performance constraints encourage \textbf{hybrid TLS 1.3
  (Kyber+ECDHE)} as interim measure.
\end{itemize}

\subsubsection{5.6.2 Healthcare}\label{healthcare}

\begin{itemize}
\tightlist
\item
  Patient records need 70‑year confidentiality.
\item
  Medical devices often lack firmware update paths → hardware refresh
  cycles must accelerate.
\end{itemize}

\subsubsection{5.6.3 Critical
infrastructure}\label{critical-infrastructure}

\begin{itemize}
\tightlist
\item
  Industrial control protocols (OPC~UA, DNP3) historically weak on
  crypto; retrofit costs high.
\item
  Quantum risk intersects safety risk → regulator scrutiny rising.
\end{itemize}

\subsection{5.7 Risk quantification
models}\label{risk-quantification-models}

\subsubsection{5.7.1 Mosca inequality}\label{mosca-inequality}

\texttt{T\_shelf‑life~+\ T\_migration~\textgreater{}~T\_threat~⇒\ exposure}
* \texttt{T\_shelf‑life} -- required confidentiality window (years) *
\texttt{T\_migration} -- time to complete PQC rollout (years) *
\texttt{T\_threat} -- forecast years until CRQC

Applying median GRI threat horizon (2034) and typical bank migration
estimate (7\,years) leaves organisations with
\textbf{\textless\,2\,years to start} if they store 10‑year confidential
data.

\subsubsection{5.7.2 Quantum‑adjusted risk score
(QARS)}\label{quantumadjusted-risk-score-qars}

Section~9 formalises QARS = \texttt{w₁·(T\_shelf/T\_threat)\ +\ …}. This
chapter seeds baseline values for \texttt{T\_threat} according to expert
surveys, and Section~9 will refine per sector.

\subsection{5.8 Emerging technical
counter‑measures}\label{emerging-technical-countermeasures}

\begin{enumerate}
\def\labelenumi{\arabic{enumi}.}
\tightlist
\item
  \textbf{Hybrid key exchange} -- IETF RFC~9399 profiles Kyber + X25519.
\item
  \textbf{Hash‑based signatures} -- SPHINCS+ for firmware where
  statelessness matters.
\item
  \textbf{Quantum‑resistant VPNs} -- WireGuard fork with Kyber prime,
  early pilots at European research networks.
\item
  \textbf{Hardware crypto‑agility} -- HSM vendors announcing firmware
  roadmaps targeting FIPS~203 Level~3 by 2026.
\end{enumerate}

\subsection{5.9 Uncertainty and accelerating
factors}\label{uncertainty-and-accelerating-factors}

\begin{longtable}[]{@{}
  >{\raggedright\arraybackslash}p{(\linewidth - 4\tabcolsep) * \real{0.3919}}
  >{\raggedright\arraybackslash}p{(\linewidth - 4\tabcolsep) * \real{0.3378}}
  >{\raggedright\arraybackslash}p{(\linewidth - 4\tabcolsep) * \real{0.2703}}@{}}
\toprule\noalign{}
\begin{minipage}[b]{\linewidth}\raggedright
Factor
\end{minipage} & \begin{minipage}[b]{\linewidth}\raggedright
Might \textbf{accelerate} CRQC
\end{minipage} & \begin{minipage}[b]{\linewidth}\raggedright
Might \textbf{delay} CRQC
\end{minipage} \\
\midrule\noalign{}
\endhead
\bottomrule\noalign{}
\endlastfoot
Error‑correction code advances & Breakthroughs in LDPC‑surface hybrids &
Diminishing returns in code discovery \\
Venture funding & Sustained VC + government subsidies & Investment
winter post‑2026 \\
Geopolitical race & State‑level moonshot funding (US, CN) & Export
controls on cryogenics \\
Hardware yields & Photonic interconnect yields improve & Cryogenic
supply‑chain bottlenecks \\
\end{longtable}

\textbf{Scenario planning} (Appendix~B) explores a ``Fast‑Track'' case
(CRQC\,=\,2029) and ``Delayed'' case (CRQC\,=\,2040) to stress‑test
organisational roadmaps.

\subsection{5.10 Key takeaways for PAREK
implementation}\label{key-takeaways-for-parek-implementation}

\begin{enumerate}
\def\labelenumi{\arabic{enumi}.}
\tightlist
\item
  \textbf{Start now} -- inventory and pilot migrations must commence by
  2026 to remain compliant with EU roadmap.
\item
  \textbf{Assume shrinkage in threat horizon} -- treat 2030 as plausible
  worst‑case, not aspirational.
\item
  \textbf{Focus on data longevity} -- prioritise assets whose
  confidentiality window extends into the 2030s.
\item
  \textbf{Engage suppliers early} -- Section~13 outlines contract
  clauses; delays compound on CRQC acceleration.
\item
  \textbf{Invest in crypto‑agility} -- architectures that can hot‑swap
  algorithms mitigate uncertainty.
\end{enumerate}

\subsection{5.11 References}\label{references}

\begin{enumerate}
\def\labelenumi{\arabic{enumi}.}
\tightlist
\item
  Global Risk Institute (2024). \emph{Quantum Threat Timeline Report
  2024}.
\item
  Global Risk Institute (2025). \emph{Quantum Threat Timeline 2025:
  Executive Perspectives}.
\item
  IBM (2024). \emph{Roadmap to Quantum‑Centric Supercomputers}.
\item
  IBM (2025). \emph{The Era of Quantum Utility Is Here}.
\item
  NIST (2024). \emph{Approval of FIPS~203,\,204,\,205}.
\item
  Keyfactor (2024). \emph{Harvest Now, Decrypt Later}.
\item
  Marlink SOC Report (2025). \emph{Harvest‑Decrypt Incidents in Maritime
  Sector}.
\item
  Europol Quantum Safe Financial Forum (2025). \emph{Recommendations to
  European Banks}.
\item
  SecurityWeek (2025). \emph{Cyber Insights 2025 -- Quantum and the
  Threat to Encryption}.
\item
  Gidney, C. \&~Ekerå, M. (2023). \emph{How to Factor 2048‑bit RSA
  Integers in 8~Hours with 20~Million Noisy Qubits}.
\end{enumerate}

\section{6 PQC Methodology}\label{pqc-methodology}

\subsection{6.1 Purpose and position of this
chapter}\label{purpose-and-position-of-this-chapter}

This methodology bridges the ``\textbf{why}'' articulated in the
quantum-threat literature with the practical ``\textbf{how}'' codified
in the PAREK framework. It supplies a repeatable
lifecycle---\emph{discover → assess → plan → execute → improve}---that
any EU organisation can embed in its security management system and map
onto the milestones of the Coordinated Implementation Roadmap
(first-steps 2026, high-risk cut-over 2030, medium-risk completion 2035)
.

\subsection{6.2 Scientific foundation}\label{scientific-foundation}

\textbf{Quantum risk geometry.} Shor's and Grover's algorithms prove
that once a \emph{cryptographically-relevant quantum computer} (CRQC)
exists, RSA/ECC and many symmetric-key constructions lose their assumed
security margins. The most widely used quantitative model is the
\emph{Mosca Inequality}:

\begin{quote}
\textbf{T­­shelf-life + T­­migration \textgreater{} T­­threat ⇒ your data will
be exposed.}
\end{quote}

The shelf-life of the data, the organisation's migration time, and the
expert-assessed CRQC timeline must be evaluated together; breaches can
begin long before a CRQC is built through
\emph{Harvest-Now-Decrypt-Later} (HNDL) attacks .

\textbf{Expert forecasts.} The 2024 Global Risk Institute survey of 32
quantum-hardware experts gives a median estimate of 11--15 years for a
CRQC able to break RSA-2048, but with a heavy tail of earlier arrivals .
Gidney \& Ekerå's resource estimate (20 million noisy qubits, 8 hours)
and subsequent error-correction progress validate that such a machine is
an engineering---rather than scientific---challenge . Because these
forecasts shift annually, the methodology demands continuous refresh of
\emph{T­­threat}.

\textbf{Standards landscape.} In August 2024 NIST issued the first three
Federal Information Processing Standards: FIPS 203 (ML-KEM / Kyber),
FIPS 204 (ML-DSA / Dilithium) and FIPS 205 (SPHINCS+) . Forthcoming FIPS
206 (BIKE) and ISO/ETSI profiles will refine parameter sets, but the
decision rule is already clear: design choices should default to these
lattice- or hash-based schemes unless an explicit profile (IoT,
constrained, statutory) dictates otherwise.

\subsection{6.3 Design principles}\label{design-principles}

\begin{enumerate}
\def\labelenumi{\arabic{enumi}.}
\tightlist
\item
  \textbf{Crypto-agility first.} Because algorithm lifetimes are
  uncertain, architectures must allow hot-swapping of primitives without
  forklift upgrades .
\item
  \textbf{Inventory before surgery.} Every migration failure studied by
  TNO traced back to an incomplete asset list; hence inventory is a
  non-negotiable gate .
\item
  \textbf{Hybrid ≥ single-stack.} Where performance permits, run
  lattice-based KEMs or signatures \emph{alongside} existing ECC/RSA
  until the latter can be fully retired. ETSI/IETF interop plug-tests
  show this halves rollback risk .
\item
  \textbf{Evidence over assertion.} Each stage outputs machine-readable
  artefacts---CBOMs, risk scores, migration run-books---that auditors
  and regulators can parse automatically.
\end{enumerate}

\subsection{6.4 Lifecycle phases}\label{lifecycle-phases}

\subsubsection{6.4.1 Phase 0 -- Programme
mobilisation}\label{phase-0-programme-mobilisation}

Although not counted among the five PAREK stages, a short mobilisation
sprint (4--6 weeks) is advisable to assign roles, secure budget and
ratify the scope statements defined in §3.

\subsubsection{6.4.2 Phase 1 -- Cryptographic discovery \& inventory
(``P'' in
PAREK)}\label{phase-1-cryptographic-discovery-inventory-p-in-parek}

\textbf{Objective.} Build a \emph{single source of truth} describing
every algorithm, key, certificate, protocol, hardware module and
crypto-library instance.

\textbf{Process.}

\begin{itemize}
\tightlist
\item
  Crawl binaries and source trees with pattern-matching and
  dynamic-analysis tools.
\item
  Enrich findings with network captures and certificate-transparency
  logs.
\item
  Normalise results into a \textbf{Cryptography Bill of Materials
  (CBOM)}---an extension of CycloneDX 1.6---which supports
  \textgreater20 asset types (algorithm, protocol, key, seed, nonce,
  etc.) .
\item
  Link CBOMs back to software SBOMs via \emph{bom-link} URNs so each
  application instance can be traced to its crypto footprint .
\end{itemize}

\textbf{Output artefacts.}

\begin{itemize}
\tightlist
\item
  CBOM JSON (one per application or shared library)
\item
  Discovery tooling report with false-positive triage
\item
  Gap register listing unscanned networks or black-box third-party
  services
\end{itemize}

\subsubsection{6.4.3 Phase 2 -- Quantum risk assessment
(``A'')}\label{phase-2-quantum-risk-assessment-a}

\textbf{Objective.} Quantify urgency and migration difficulty, then
classify systems into EU ``high / medium / low'' buckets.

\textbf{Scoring model.} Extend Mosca's inequality into a composite
\emph{Quantum-Adjusted Risk Score (QARS)}:

\begin{verbatim}
QARS = w1·(T_shelf-life / T_threat) + w2·(Migration_Cost / CapEx_Budget)
      + w3·(Data_Sensitivity) + w4·(Exposure_Surface)
\end{verbatim}

where weights \emph{w1--w4} are calibrated by sector regulators. TNO's
handbook suggests default weightings of 0.35 / 0.25 / 0.25 / 0.15 after
pilot workshops .

\textbf{Scientific reference.} Mosca \& Mulholland's original risk
methodology underpins the formula and justifies linear aggregation of
shelf-life and migration vectors .

\textbf{Output artefacts.}

\begin{itemize}
\tightlist
\item
  Per-system risk dossier (QARS, assumptions, reviewer sign-off)
\item
  Heat-map dashboard for C-suite and board reporting
\end{itemize}

\subsubsection{6.4.4 Phase 3 -- Road-mapping \& readiness planning
(``R'')}\label{phase-3-road-mapping-readiness-planning-r}

\textbf{Objective.} Translate scores into dated, budgeted work-packages
aligned with EU milestones.

\textbf{Steps.}

\begin{enumerate}
\def\labelenumi{\arabic{enumi}.}
\tightlist
\item
  \textbf{Prioritise} systems with QARS ≥ 0.65 for immediate pilot
  migrations.
\item
  \textbf{Allocate buffer time} for external dependencies (e.g.,
  hardware security modules awaiting FIPS 203 validation).
\item
  \textbf{Sequence pilots} to maximise knowledge reuse---start with a
  low-volume API gateway, then propagate the playbook to high-volume
  payment stacks.
\item
  \textbf{Integrate supplier clauses} requiring CBOM delivery and
  PQC-ready firmware by 2028 for high-risk contracts .
\end{enumerate}

\textbf{Output artefacts.}

\begin{itemize}
\tightlist
\item
  Gantt chart or kanban milestones
\item
  Budget breakdown: licences, hardware refresh, training, contingency
\item
  Contract addenda language for suppliers
\end{itemize}

\subsubsection{6.4.5 Phase 4 -- Execution \& migration
(``E'')}\label{phase-4-execution-migration-e}

\textbf{Objective.} Replace---or wrap in hybrid mode---all
quantum-vulnerable primitives, while preserving service levels.

\textbf{Preferred migration patterns (scientific rationale in
brackets):}

\begin{itemize}
\tightlist
\item
  \textbf{Kyber-in-TLS 1.3 hybrid KEM:} Adds \textless2 kB to handshake;
  end-to-end field results show negligible latency increase at sub-10 ms
  RTT .
\item
  \textbf{Dilithium signatures for code signing:} Larger certificates
  (\textasciitilde14 kB) but verified 100× faster than SPHINCS+, making
  it fit for CI/CD pipelines.
\item
  \textbf{SPHINCS+ for long-life artefacts (firmware, legal archives):}
  Stateless hash-based design offers security with minimal cryptanalysis
  uncertainty.
\item
  \textbf{Double-wrap archives:} Encrypt once with AES-256-GCM, then
  wrap the symmetric key via Kyber or BIKE to separate confidentiality
  from PQC adoption pace.
\end{itemize}

\textbf{Change-control safeguards.} Each rollout includes a
cryptographic \emph{canary test}, real-time telemetry on handshake
success rates, and a rollback plan tied to traffic shadowing.

\textbf{Output artefacts.}

\begin{itemize}
\tightlist
\item
  Migration run-books and playbooks per platform
\item
  Performance-impact report versus baseline
\item
  Certificate revocation \& renewal schedule
\end{itemize}

\subsubsection{6.4.6 Phase 5 -- Key-governance \& continuous improvement
(``K'')}\label{phase-5-key-governance-continuous-improvement-k}

\textbf{Objective.} Ensure that once migrated, systems stay
quantum-resilient---even as algorithms evolve or new vulnerabilities
surface.

\textbf{Controls.}

\begin{itemize}
\tightlist
\item
  \textbf{Continuous CBOM scanning:} Weekly delta scans detect drift;
  policy engines flag any newly imported RSA/ECC library versions .
\item
  \textbf{Policy attestation via CycloneDX Attestations:} Suppliers
  attach machine-readable claims linking binaries to NIST/FIPS
  conformity, automating compliance checks .
\item
  \textbf{Crypto-agility playbooks:} Design patterns
  (algorithm-independent keystores, versioned protocol negotiation,
  feature flags) enable hot re-parametrisation---a requirement
  emphasised by NCSC-NL and echoed in TNO Step 4.4 .
\item
  \textbf{Metric suite:} Mean Time To Remediate Weak Crypto (MTTR-C), \%
  assets with valid CBOM, \% PQC certificates in production. These feed
  into ENISA reporting and, under NIS-2, into supervisory audits.
\end{itemize}

\subsection{6.5 Embedding the methodology in EU
governance}\label{embedding-the-methodology-in-eu-governance}

Member States' NIS Cooperation Group work-stream recommends each
national roadmap publish quarterly status against the core measures
above and contribute pilot results to the EU testing infrastructure . By
harmonising metrics and artefacts (CBOM JSON, QARS spreadsheets, FIPS
certificate IDs), the methodology enables cross-border comparability and
pooled threat-intelligence.

\subsection{6.6 Limitations and future
research}\label{limitations-and-future-research}

While lattice-based KEMs currently lead standardisation, code-based
(Classic McEliece) and isogeny-based (SIKE-like) schemes deserve niche
consideration; the methodology therefore reserves an \emph{Experimental
Track} for low-volume prototypes. The scientific community is still
refining fault-tolerance thresholds---e.g., the debate around
``dynamic-logical qubits'' may shift T­­threat earlier or later.
Organisations must budget for annual model recalibration as these
estimates mature.

\subsection{6.7 Conclusion}\label{conclusion}

This PQC Methodology equips EU organisations with a science-grounded,
regulator-aligned and audit-ready pathway from cryptographic discovery
to long-term quantum resilience. By anchoring every decision in
measurable artefacts---CBOMs, risk scores, migration run-books---and
iterating through the PAREK lifecycle, enterprises can defend today's
and tomorrow's data against the quantum horizon.

\section{7 Framework Overview}\label{framework-overview}

\begin{quote}
\textbf{Purpose of this chapter} -- give readers a \emph{one‑stop}
visual and narrative tour of the PAREK Framework, explaining how its
five stages interlock, what artefacts they exchange, and how the cycle
repeats to deliver continuous crypto‑agility.
\end{quote}

\subsection{7.1 At‑a‑glance diagram}\label{ataglance-diagram}

\begin{verbatim}
┌────────┐   Discover     ┌─────────────┐   Quantify     ┌───────────┐   Plan     ┌─────────┐  Deploy     ┌────────────┐  Monitor
│   P    │──────────────►│      A      │──────────────►│     R     │──────────►│    E    │────────────►│      K     │──┐
│ Asset  │   CBOM JSON    │  Risk Score │   QARS heatmap │ Road‑map  │  Gantt      │ Migration│  Telemetry   │  Metrics   │  │
│  Inv   │◄──────────────│             │◄──────────────│           │◄──────────│          │◄────────────│           │◄─┘
└────────┘   Gaps & todo   └─────────────┘   Feedback     └───────────┘  Budget     └─────────┘  Incidents   └────────────┘
\end{verbatim}

\emph{Figure\,1\,--\,PAREK Framework life‑cycle (high‑level data flow)}

\subsection{7.2 Stage synopses}\label{stage-synopses}

\subsubsection{\texorpdfstring{7.2.1 \textbf{P~-- Post‑Quantum Asset \&
Algorithm
Inventory}}{7.2.1 P~-- Post‑Quantum Asset \& Algorithm Inventory}}\label{p-postquantum-asset-algorithm-inventory}

\emph{Goal} -- create a machine‑readable \textbf{Cryptography Bill of
Materials (CBOM)} for every software, hardware and service component.
Uses automated scanners, manual surveys and supplier attestations.
Output feeds directly into Stage~A.

\subsubsection{\texorpdfstring{7.2.2 \textbf{A~-- Assessment of Quantum
Risk}}{7.2.2 A~-- Assessment of Quantum Risk}}\label{a-assessment-of-quantum-risk}

\emph{Goal} -- compute a \textbf{Quantum‑Adjusted Risk Score (QARS)} for
each asset by blending data shelf‑life, migration effort and CRQC
timeline inputs. High‑risk items graduate to Stage~R while low‑risk
items loop back for periodic re‑assessment.

\subsubsection{\texorpdfstring{7.2.3 \textbf{R~-- Road‑mapping \&
Readiness
Planning}}{7.2.3 R~-- Road‑mapping \& Readiness Planning}}\label{r-roadmapping-readiness-planning}

\emph{Goal} -- translate scores into a time‑phased, resourced roadmap
aligned to EU milestones (2026‑2035). Outputs Gantt charts, budget
forecasts, and supplier alignment plans. Detailed in §10.

\subsubsection{\texorpdfstring{7.2.4 \textbf{E~-- Execution \&
Migration}}{7.2.4 E~-- Execution \& Migration}}\label{e-execution-migration}

\emph{Goal} -- deploy PQC or hybrid primitives using controlled
roll‑outs, rollback strategies and performance monitoring. Produces
migration run‑books and incident telemetry.

\subsubsection{\texorpdfstring{7.2.5 \textbf{K~-- Key‑Governance \&
Continuous
Improvement}}{7.2.5 K~-- Key‑Governance \& Continuous Improvement}}\label{k-keygovernance-continuous-improvement}

\emph{Goal} -- sustain crypto‑agility through continuous scanning,
supplier attestations, KPIs and policy refreshes. Feeds new discoveries
back to Stage~P, closing the loop.

\subsection{7.3 Artefact hand‑offs}\label{artefact-handoffs}

\begin{longtable}[]{@{}
  >{\raggedright\arraybackslash}p{(\linewidth - 6\tabcolsep) * \real{0.0900}}
  >{\raggedright\arraybackslash}p{(\linewidth - 6\tabcolsep) * \real{0.3800}}
  >{\raggedright\arraybackslash}p{(\linewidth - 6\tabcolsep) * \real{0.2100}}
  >{\raggedright\arraybackslash}p{(\linewidth - 6\tabcolsep) * \real{0.3200}}@{}}
\toprule\noalign{}
\begin{minipage}[b]{\linewidth}\raggedright
From ▶ To
\end{minipage} & \begin{minipage}[b]{\linewidth}\raggedright
Artefact
\end{minipage} & \begin{minipage}[b]{\linewidth}\raggedright
Format
\end{minipage} & \begin{minipage}[b]{\linewidth}\raggedright
Purpose
\end{minipage} \\
\midrule\noalign{}
\endhead
\bottomrule\noalign{}
\endlastfoot
P ▶ A & CBOMs (per system) & CycloneDX~JSON + sig & Input for risk
scoring \\
A ▶ R & QARS risk registry & CSV / Grafana feed & Prioritise backlog \\
R ▶ E & Work‑package definitions & Jira Epics~+ Stories & Guide
migration teams \\
E ▶ K & Deployment telemetry, incident reports & Prometheus / GRC logs &
Measure success, trigger alerts \\
K ▶ P & Revised asset list, KPIs & JSON diff, dashboard & Refresh
inventory \& repeat cycle \\
\end{longtable}

\subsection{7.4 Governance layers}\label{governance-layers}

\subsubsection{7.4.1 Strategic layer}\label{strategic-layer}

\emph{PAREK Steering Committee} (quarterly) endorses roadmaps, budget,
and policy changes.

\subsubsection{7.4.2 Operational layer}\label{operational-layer}

\emph{Crypto Working Group} (monthly) coordinates cross‑team
dependencies, tooling upgrades and incident response.

\subsubsection{7.4.3 Tactical layer}\label{tactical-layer}

Dedicated \emph{Migration Squads} execute Jira stories, report blockers
and feed metrics to dashboards.

\subsection{7.5 Alignment with PQC Methodology
(§6)}\label{alignment-with-pqc-methodology-6}

Section~6 introduces the
\textbf{discover\,→\,assess\,→\,plan\,→\,execute\,→\,improve} cycle at
conceptual level. This chapter grounds that abstract model in concrete
artefacts, roles and data flows, forming the \emph{Rosetta Stone} that
maps theory to practice.

Key alignment points:

\begin{itemize}
\tightlist
\item
  \textbf{Inventory before surgery} principle manifests as the strict P
  ▶ A gate.
\item
  \textbf{Crypto‑agility first} translates into K~metrics (\# low‑risk
  ECDSA certs trending~→~0).
\end{itemize}

\subsection{7.6 Integration with supply‑chain
(§13)}\label{integration-with-supplychain-13}

Supplier CBOMs are imported into Stage~P; supplier roadmaps and
compliance clauses sit in Stage~R; supplier attestation SLAs are
monitored in Stage~K. Thus, the framework treats third‑party components
as \emph{co‑equal citizens} in the life‑cycle.

\subsection{7.7 Quality gates \& escalation
paths}\label{quality-gates-escalation-paths}

\begin{verbatim}
[Gate G1] CBOM coverage ≥ 95 % —► proceed to Stage A  | else: raise Inventory CAPA
[Gate G2] QARS sign‑off by CISO —► proceed to Stage R  | else: re‑score anomalies
[Gate G3] Budget approval —► proceed to Stage E        | else: escalate to CFO
[Gate G4] KPI trend green 3 months —► close loop       | else: open incident review
\end{verbatim}

Each gate has an \emph{owner}, \emph{entry criteria} and \emph{exit
criteria}, ensuring accountability.

\subsection{7.8 Toolchain reference
stack}\label{toolchain-reference-stack}

\begin{longtable}[]{@{}
  >{\raggedright\arraybackslash}p{(\linewidth - 4\tabcolsep) * \real{0.0633}}
  >{\raggedright\arraybackslash}p{(\linewidth - 4\tabcolsep) * \real{0.4937}}
  >{\raggedright\arraybackslash}p{(\linewidth - 4\tabcolsep) * \real{0.4430}}@{}}
\toprule\noalign{}
\begin{minipage}[b]{\linewidth}\raggedright
Stage
\end{minipage} & \begin{minipage}[b]{\linewidth}\raggedright
Open‑source baseline tools
\end{minipage} & \begin{minipage}[b]{\linewidth}\raggedright
Commercial alternatives
\end{minipage} \\
\midrule\noalign{}
\endhead
\bottomrule\noalign{}
\endlastfoot
P & \textbf{oqs‑scanner}, \texttt{cyclonedx‑python‑lib} & Venafi TLS
Protect, Fortanix DSM \\
A & \texttt{pandas~+~risk‑calc.py} & RSA Archer, ServiceNow VRM \\
R & \texttt{ganttlab}, GitLab Road‑maps & Atlassian Advanced Roadmaps \\
E & \texttt{openssl‑oqs}, QEMU testbed & Entrust nShield, Thales
CipherTrust \\
K & Grafana, Prometheus & Splunk ES, Elastic SIEM \\
\end{longtable}

A Terraform module (\texttt{scripts/terraform/parek‑stack.tf})
provisions the open‑source stack for pilots.

\subsection{7.9 Maturity model}\label{maturity-model}

\begin{longtable}[]{@{}
  >{\raggedright\arraybackslash}p{(\linewidth - 4\tabcolsep) * \real{0.2105}}
  >{\raggedright\arraybackslash}p{(\linewidth - 4\tabcolsep) * \real{0.5053}}
  >{\raggedright\arraybackslash}p{(\linewidth - 4\tabcolsep) * \real{0.2842}}@{}}
\toprule\noalign{}
\begin{minipage}[b]{\linewidth}\raggedright
Level
\end{minipage} & \begin{minipage}[b]{\linewidth}\raggedright
Characteristics
\end{minipage} & \begin{minipage}[b]{\linewidth}\raggedright
Typical KPI values
\end{minipage} \\
\midrule\noalign{}
\endhead
\bottomrule\noalign{}
\endlastfoot
\textbf{1 -- Ad~hoc} & No CBOM, PQC unknown, vendors unmanaged &
SC‑1~\textless\,10\,\% \\
\textbf{2 -- Defined} & Static inventory, pilot QARS, roadmap draft &
SC‑1≈50\,\%, QARS~cov.~≥\,30\,\% \\
\textbf{3 -- Managed} & Approved roadmap, hybrid pilots in prod &
SC‑1≥90\,\%, KPI trend ↑ \\
\textbf{4 -- Quantitative} & Real‑time metrics, full PQC for high‑risk
assets & KPI SLA ≤\,5\,\% viol. \\
\textbf{5 -- Optimising} & Continuous crypto‑agility, auto‑rotation &
Zero unsupported algs \\
\end{longtable}

Stage~K owns the maturity assessment and reports progression each
quarter.

\subsection{7.10 Next steps for readers}\label{next-steps-for-readers}

\begin{itemize}
\tightlist
\item
  \textbf{Architects} -- dive into §8--12 for deep‑dive guidance per
  stage.
\item
  \textbf{Project managers} -- reference §10 for detailed timelines and
  resource models.
\item
  \textbf{Suppliers} -- jump to §13 for contract clauses and CBOM spec
  requirements.
\end{itemize}

\section{8 P -- Post‑Quantum Asset \& Algorithm
Inventory}\label{p-postquantum-asset-algorithm-inventory-1}

\begin{quote}
\textbf{Purpose} -- provide a succinct overview of how to catalogue all
cryptographic assets and algorithms so that subsequent risk scoring
(Stage~A) is based on complete, reliable data. This short version is
intended as a quick‑start guide; a full procedural manual will follow in
v0.2.
\end{quote}

\subsection{8.1 What is a CBOM?}\label{what-is-a-cbom}

A \textbf{Cryptography Bill of Materials (CBOM)} is a machine‑readable
inventory (CycloneDX JSON) listing every algorithm, key, certificate,
protocol and crypto‑module used by a software, hardware or service
component. Think of it as an SBOM for cryptography.

\subsection{8.2 Minimal discovery
workflow}\label{minimal-discovery-workflow}

\begin{longtable}[]{@{}
  >{\raggedright\arraybackslash}p{(\linewidth - 6\tabcolsep) * \real{0.0635}}
  >{\raggedright\arraybackslash}p{(\linewidth - 6\tabcolsep) * \real{0.4762}}
  >{\raggedright\arraybackslash}p{(\linewidth - 6\tabcolsep) * \real{0.3651}}
  >{\raggedright\arraybackslash}p{(\linewidth - 6\tabcolsep) * \real{0.0952}}@{}}
\toprule\noalign{}
\begin{minipage}[b]{\linewidth}\raggedright
Step
\end{minipage} & \begin{minipage}[b]{\linewidth}\raggedright
Action
\end{minipage} & \begin{minipage}[b]{\linewidth}\raggedright
Tool / Source
\end{minipage} & \begin{minipage}[b]{\linewidth}\raggedright
Output
\end{minipage} \\
\midrule\noalign{}
\endhead
\bottomrule\noalign{}
\endlastfoot
1 & Binary \& source code scan & \texttt{oqs‑scanner}, regex & Algorithm
list \\
2 & Network traffic sampling & Zeek, Wireshark & Cipher‑suite map \\
3 & Certificate inventory & CT~logs, PKI DB & x509 dump \\
4 & Supplier CBOM ingest & API / S3 / email (§13) & External JSON \\
5 & Manual survey for edge assets & Google~Forms & Gap register \\
\end{longtable}

Run steps 1--4 in parallel; perform step~5 only if coverage
\textless\,95\,\%.

\subsection{8.3 Essential data fields}\label{essential-data-fields}

\begin{enumerate}
\def\labelenumi{\arabic{enumi}.}
\tightlist
\item
  \texttt{algorithm} -- e.g., \texttt{rsa2048}, \texttt{ml‑kem‑768}\\
\item
  \texttt{context} -- \texttt{tls1.3}, \texttt{ssh2}, \texttt{jwt}\\
\item
  \texttt{keySize} / \texttt{parameterSet}\\
\item
  \texttt{usage} -- signing, encryption, key agreement\\
\item
  \texttt{expires} -- ISO~date for certificates/keys\\
\item
  \texttt{hardwareAnchor} -- HSM/TPM model + firmware
\end{enumerate}

Include a \texttt{scanTimestamp} and digital signature
(\texttt{*.cbom.sig}).

\subsection{8.4 Quality gate G1 (inventory
lock)}\label{quality-gate-g1-inventory-lock}

\begin{itemize}
\tightlist
\item
  \textbf{Metric} -- \% assets with valid CBOM ≥\,95\,\%.
\item
  \textbf{Owner} -- Asset‑Inventory Lead.
\item
  \textbf{Tool} -- Grafana dashboard \texttt{CBOM‑coverage}.
\end{itemize}

If coverage \textless\,95\,\%, raise Corrective Action Plan and block
Stage~A.

\subsection{8.5 Outputs}\label{outputs}

\begin{itemize}
\tightlist
\item
  \textbf{CBOM repository} -- \texttt{git@repo:cbom/} with one JSON per
  system.
\item
  \textbf{Gap register} -- CSV of unscanned or unknown assets.
\item
  \textbf{Coverage dashboard} -- auto‑refreshed via Prometheus.
\end{itemize}

\subsection{8.6 Common pitfalls}\label{common-pitfalls}

\begin{enumerate}
\def\labelenumi{\arabic{enumi}.}
\tightlist
\item
  \textbf{Duplicate asset IDs} -- enforce UUID naming.
\item
  \textbf{Missing hardware mapping} -- integrate CMDB export.
\item
  \textbf{Supplier lag} -- tie CBOM submission to invoice milestone.
\end{enumerate}

\subsection{8.7 Next steps}\label{next-steps}

Once G1 is passed, hand off the consolidated CBOM set to Stage~A for
QARS calculation. Retain automated nightly scans to catch drift.

\section{9 A -- Assessment of Quantum
Risk}\label{a-assessment-of-quantum-risk-1}

\begin{quote}
\textbf{Purpose of this chapter} -- define a repeatable, data‑driven
methodology for quantifying how urgently each system, data set or
supplier must migrate to post‑quantum cryptography. The output of this
stage---\textbf{Quantum‑Adjusted Risk Scores (QARS)}---feeds
Road‑mapping (§10) and underpins budget and resource prioritisation.
\end{quote}

\subsection{9.1 Why risk scoring
matters}\label{why-risk-scoring-matters}

Cryptographic migration budgets are finite, systems are heterogeneous,
and CRQC timelines are uncertain. A robust risk model ensures that
\emph{business‑critical, long‑lived} data is protected first, while
low‑impact assets follow a just‑in‑time trajectory. Without
quantification, organisations either under‑invest (and face
data‑exposure liability) or over‑invest (and stall other security
priorities). The QARS model harmonises quantitative (years, euros,
CVSS‑scores) and qualitative (business impact, regulatory penalty)
inputs into a single comparable metric.

\subsection{9.2 Inputs and
prerequisites}\label{inputs-and-prerequisites}

\begin{longtable}[]{@{}
  >{\raggedright\arraybackslash}p{(\linewidth - 4\tabcolsep) * \real{0.4868}}
  >{\raggedright\arraybackslash}p{(\linewidth - 4\tabcolsep) * \real{0.3158}}
  >{\raggedright\arraybackslash}p{(\linewidth - 4\tabcolsep) * \real{0.1974}}@{}}
\toprule\noalign{}
\begin{minipage}[b]{\linewidth}\raggedright
Input
\end{minipage} & \begin{minipage}[b]{\linewidth}\raggedright
Source
\end{minipage} & \begin{minipage}[b]{\linewidth}\raggedright
Refresh cadence
\end{minipage} \\
\midrule\noalign{}
\endhead
\bottomrule\noalign{}
\endlastfoot
Cryptography Bill of Materials (CBOM) & Stage~P (§8) & Nightly \\
Data‑classification registry & GDPR/NIS‑2 policy owners & Quarterly \\
CRQC threat horizon (\texttt{T\_threat}) & §5 + external forecasts &
Annual + ad~hoc \\
Migration effort estimates & Architecture \& dev‑ops & Sprintly \\
Exposure Surface Index (ESI) & Pentest/vuln‑scan teams & Monthly \\
\end{longtable}

If any CBOM asset lacks a data‑classification tag or migration estimate,
it is flagged ``information incomplete'' and cannot be scored until gaps
are resolved.

\subsection{9.3 The QARS formula}\label{the-qars-formula}

PAREK extends Mosca's inequality into a \textbf{multi‑factor linear
model}:

\begin{verbatim}
QARS = w₁·(T_shelf / T_threat) + w₂·(T_migration / T_buffer) + w₃·Data_Sensitivity +
       w₄·Exposure_Surface + w₅·Compliance_Penalty
\end{verbatim}

\begin{longtable}[]{@{}
  >{\raggedright\arraybackslash}p{(\linewidth - 4\tabcolsep) * \real{0.2857}}
  >{\raggedright\arraybackslash}p{(\linewidth - 4\tabcolsep) * \real{0.6429}}
  >{\raggedright\arraybackslash}p{(\linewidth - 4\tabcolsep) * \real{0.0714}}@{}}
\toprule\noalign{}
\begin{minipage}[b]{\linewidth}\raggedright
Symbol
\end{minipage} & \begin{minipage}[b]{\linewidth}\raggedright
Explanation
\end{minipage} & \begin{minipage}[b]{\linewidth}\raggedright
Scale
\end{minipage} \\
\midrule\noalign{}
\endhead
\bottomrule\noalign{}
\endlastfoot
\texttt{T\_shelf} & Required confidentiality window (years) & 0--25+ \\
\texttt{T\_threat} & Forecast years until CRQC (default~=~9--15) &
0--20 \\
\texttt{T\_migration} & Estimated time to complete PQC rollout (yrs) &
0--5 \\
\texttt{T\_buffer} & Policy‑set buffer (yrs, default~=~2) & fixed \\
\texttt{Data\_Sensitivity} & GDPR Level~1‑3 or internal A‑E scale &
0.1‑1 \\
\texttt{Exposure\_Surface} & Normalised count of public endpoints \&
CVEs & 0‑1 \\
\texttt{Compliance\_Penalty} & 0.2 if asset falls under NIS‑2 critical
infra & 0/0.2 \\
\end{longtable}

Weights \texttt{w₁…w₅} default to
\textbf{0.30\,/\,0.20\,/\,0.25\,/\,0.15\,/\,0.10} but can be re‑tuned at
sector level (e.g., finance may increase sensitivity weight to 0.35).
QARS outputs a \textbf{unitless value between 0 and 1} where
≥\,0.65~=~\emph{high}, 0.35--0.64~=~\emph{medium},
\textless\,0.35~=~\emph{low}.

\subsection{9.4 Data‑collection pipeline}\label{datacollection-pipeline}

\begin{enumerate}
\def\labelenumi{\arabic{enumi}.}
\tightlist
\item
  \textbf{Ingest} -- nightly ETL job pulls CBOM JSON, joins CMDB IDs,
  merges data‑classification tags.
\item
  \textbf{Enrich} -- scrape CVE feeds (NVD) to calculate Exposure
  Surface Index for IP addresses/certs.
\item
  \textbf{Estimate} -- dev‑ops provides story‑point‑based migration
  effort which converts to months via team velocity.
\item
  \textbf{Compute} -- Python microservice
  (\texttt{scripts/qars\_calc.py}) applies formula; outputs per‑asset
  records to Postgres and Grafana.
\item
  \textbf{Validate} -- security architects review anomalies (e.g.,
  QARS\,\textgreater\,0.9 for ``low'' data system) via Jira workflow.
\end{enumerate}

All stages run in a dedicated Kubernetes namespace with signed container
images; audit logs export to Splunk for regulator access.

\subsection{9.5 Visualising risk}\label{visualising-risk}

Default dashboards include:

\begin{itemize}
\tightlist
\item
  \textbf{Heat‑map} -- assets on x‑axis (systems) vs.~y‑axis (QARS);
  colour gradient highlights urgency.
\item
  \textbf{Scatter} -- \texttt{T\_shelf} on x, \texttt{T\_migration} on
  y; diagonal line shows Mosca boundary.
\item
  \textbf{Burndown} -- number of high‑risk assets over time; target
  trend~=~zero by Q4~2030.
\end{itemize}

Grafana JSON for these panels is stored under
\texttt{assets/grafana/qars\_dash.json}.

\subsection{9.6 Quality gate G2 -- QARS
sign‑off}\label{quality-gate-g2-qars-signoff}

Before Stage~R can commence, the CISO (or delegate) must approve:

\begin{enumerate}
\def\labelenumi{\arabic{enumi}.}
\tightlist
\item
  \textbf{Coverage} -- ≥\,90\,\% of in‑scope assets have non‑null QARS.
\item
  \textbf{Accuracy} -- sample audit (10\,\%) shows \textless\,5\,\%
  variance between estimated and observed parameters.
\item
  \textbf{Documentation} -- methodology, weight settings, data sources
  captured in Confluence page \texttt{PQC/Risk‑Method}.
\end{enumerate}

Failure to meet criteria pauses migration planning; corrective actions
logged in the \emph{Risk Management} Jira project.

\subsection{9.7 Scenario analysis}\label{scenario-analysis}

PAREK requires bi‑annual \textbf{scenario runs} to test sensitivity:

\begin{longtable}[]{@{}
  >{\raggedright\arraybackslash}p{(\linewidth - 6\tabcolsep) * \real{0.1856}}
  >{\raggedright\arraybackslash}p{(\linewidth - 6\tabcolsep) * \real{0.2165}}
  >{\raggedright\arraybackslash}p{(\linewidth - 6\tabcolsep) * \real{0.2474}}
  >{\raggedright\arraybackslash}p{(\linewidth - 6\tabcolsep) * \real{0.3505}}@{}}
\toprule\noalign{}
\begin{minipage}[b]{\linewidth}\raggedright
Scenario ID
\end{minipage} & \begin{minipage}[b]{\linewidth}\raggedright
\texttt{T\_threat} assumption
\end{minipage} & \begin{minipage}[b]{\linewidth}\raggedright
Outcome metric
\end{minipage} & \begin{minipage}[b]{\linewidth}\raggedright
Implication
\end{minipage} \\
\midrule\noalign{}
\endhead
\bottomrule\noalign{}
\endlastfoot
\textbf{S‑A (Fast)} & 5~yrs (2029) & ΔHigh‑risk assets~+~35\,\% & Budget
re‑prioritisation needed \\
\textbf{S‑B (Baseline)} & Median 9~yrs (2034) & n/a & Reference
roadmap \\
\textbf{S‑C (Delayed)} & 15~yrs (2040) & ΔBudget~--~18\,\% & Optional
slow‑track for low assets \\
\end{longtable}

Results feed the CFO's risk‑adjusted cost model; Stage~10 picks
whichever roadmap keeps high‑risk completion \textless=~3\,yrs before
\texttt{T\_threat}.

\subsection{9.8 Integration with supplier
risk}\label{integration-with-supplier-risk}

Supplier CBOMs (Tier~1 \&~2) receive QARS as well. Additional factor
\texttt{Supplier\_Maturity} (scale~0--0.2) reduces QARS if vendor
demonstrates crypto‑agility lab results. Non‑compliant suppliers
auto‑escalate to \emph{Supplier Risk Queue} (§13) and may face contract
penalties.

\subsection{9.9 Common pitfalls \& mitigations
(≈\,100~words)}\label{common-pitfalls-mitigations-100-words}

\begin{enumerate}
\def\labelenumi{\arabic{enumi}.}
\tightlist
\item
  \textbf{Stale data} -- automate nightly refresh; raise alert if CBOM
  timestamp \textgreater\,7~days.
\item
  \textbf{Weight gaming} -- lock weights quarter‑by‑quarter; require
  Steering approval for changes.
\item
  \textbf{False precision} -- present score bands (high/med/low) to
  execs, not raw decimals.
\item
  \textbf{Blind spots} -- add ``Unknown'' category and track reduction
  KPI.
\end{enumerate}

\subsection{9.10 Outputs}\label{outputs-1}

\begin{itemize}
\tightlist
\item
  \texttt{risk\_registry.csv} -- asset‑level QARS, drivers, timestamp.
\item
  Grafana dashboard -- URL \texttt{/d/qars/quantum‑risk}.
\item
  Executive slide deck template
  (\texttt{assets/templates/qars‑brief.pptx}).
\end{itemize}

\subsection{9.11 Next steps}\label{next-steps-1}

Hand off risk‑registry to Stage~R for roadmap planning. Schedule next
scenario analysis within 6~months or sooner if IBM announces ≥\,1\,M
qubits.

\subsection{9.12 References}\label{references-1}

\begin{enumerate}
\def\labelenumi{\arabic{enumi}.}
\tightlist
\item
  Mosca, M. (2023). \emph{Risk Framework for Quantum Threats}.
\item
  Global Risk Institute (2024). \emph{Quantum Threat Timeline Report}.
\item
  ENISA (2024). \emph{Good Practices for Supply‑chain Risk Management}.
\item
  NIST (2024). \emph{Post‑Quantum Cryptography FIPS~203‑205}.
\item
  IBM (2025). \emph{Quantum Roadmap}.
\end{enumerate}

\section{10 R~--~Road‑mapping~\& Readiness
Planning}\label{r-roadmapping-readiness-planning-1}

\begin{quote}
\textbf{Purpose of this chapter} -- translate the quantitative urgency
produced in \emph{§9~Assessment of Quantum Risk} into a resourced, dated
and regulator‑aligned action plan that will deliver quantum resilience
across the organisation's entire technology estate.
\end{quote}

\subsection{10.1~~Scope and positioning}\label{scope-and-positioning}

Road‑mapping and readiness planning (the \textbf{R} in \emph{P‑A‑R‑E‑K})
is the linchpin phase that turns analytical findings into concrete,
executive‑approved commitments. It spans three macro tasks:

\begin{enumerate}
\def\labelenumi{\arabic{enumi}.}
\tightlist
\item
  \textbf{Prioritise} -- decide which systems, business services and
  supply‑chain partners move first, based on QARS scores, EU risk
  categories and practical constraints;
\item
  \textbf{Plan} -- build a realistic programme schedule with phased
  deployments, governance checkpoints and budget allocations; and
\item
  \textbf{Prepare} -- ensure that people, processes and technology are
  in place when execution starts (tooling, contracts, training, fallback
  strategies).
\end{enumerate}

The output is a \emph{single authoritative roadmap} that boards,
regulators and suppliers can cite. Without it, migration efforts
splinter into ad hoc projects that stall or overrun.

\subsection{10.2~~Key inputs}\label{key-inputs}

\begin{longtable}[]{@{}
  >{\raggedright\arraybackslash}p{(\linewidth - 4\tabcolsep) * \real{0.2564}}
  >{\raggedright\arraybackslash}p{(\linewidth - 4\tabcolsep) * \real{0.4103}}
  >{\raggedright\arraybackslash}p{(\linewidth - 4\tabcolsep) * \real{0.3333}}@{}}
\toprule\noalign{}
\begin{minipage}[b]{\linewidth}\raggedright
Artefact
\end{minipage} & \begin{minipage}[b]{\linewidth}\raggedright
Source section
\end{minipage} & \begin{minipage}[b]{\linewidth}\raggedright
Description
\end{minipage} \\
\midrule\noalign{}
\endhead
\bottomrule\noalign{}
\endlastfoot
\textbf{CBOM inventory} & §8 & Machine‑readable list of algorithms, keys
and protocols per system \\
\textbf{QARS scores} & §9 & Composite urgency rating (0--1) per
system/service \\
\textbf{EU roadmap milestones} & External & 2026 early pilots, 2030
high‑risk cut‑over, 2035 medium‑risk completion \\
\textbf{Budget envelope} & CFO & Multi‑year capital \& operational
funding ceiling \\
\textbf{Resource capacity} & HR / PMO & Available FTEs, external
consultants, supplier bandwidth \\
\end{longtable}

\subsection{10.3~~Process overview}\label{process-overview}

\begin{verbatim}
┌───────────────┐   1   ┌──────────────────┐     2     ┌────────────────┐   3   ┌──────────────────┐
│  Risk & Asset │──►──►│  Prioritisation  │───►───►──►│  Road‑map Plan  │──►──►│   Readiness Set   │
│   Intelligence│       └──────────────────┘           └────────────────┘       └──────────────────┘
    (§8 & §9)             (10.4)                           (10.5)                   (10.6)
\end{verbatim}

Each arrow represents a \emph{quality gate} -- the roadmap cannot
progress until the preceding artefacts are baseline‑approved.

\subsection{10.4~~Step~1~~Prioritisation
(4\,weeks)}\label{step-1-prioritisation-4-weeks}

The goal is an \textbf{ordered backlog} of migration work‑packages.

\subsubsection{10.4.1~~Segmentation}\label{segmentation}

\begin{itemize}
\tightlist
\item
  \textbf{Risk bucket} -- map QARS ≥\,0.65 to \emph{high}, 0.35--0.64 to
  \emph{medium}, \textless\,0.35 to \emph{low}.
\item
  \textbf{Business criticality} -- overlay impact tiers
  (mission‑critical, regulatory, customer‑facing, internal).
\item
  \textbf{Dependency heat‑map} -- identify technical couplings (shared
  crypto libraries, common PKI roots, hardware modules).
\end{itemize}

\subsubsection{10.4.2~~Scoring matrix}\label{scoring-matrix}

Create a \emph{Prioritisation Index (PI)} =
\texttt{w\_risk\ ×\ QARS\ +\ w\_imp\ ×\ Impact\ +\ w\_dep\ ×\ Coupling}.
Default weights 0.4 / 0.4 / 0.2 can be tuned in steering committee.

\subsubsection{10.4.3~~Pilot selection}\label{pilot-selection}

Select at least one representative workload in each domain (web, mobile,
embedded, data‑at‑rest) to validate migration playbooks. Early pilots
should have: * ≤\,10\,k TPS (to limit blast radius) * Dedicated dev‑ops
pipeline for rapid iteration * Supportive product owner

\subsection{10.5~~Step~2~~Road‑map planning
(6\,weeks)}\label{step-2-roadmap-planning-6-weeks}

This step converts the ordered backlog into a \textbf{multi‑year,
resource‑levelled Gantt}.

\subsubsection{10.5.1~~Timeline alignment}\label{timeline-alignment}

\begin{longtable}[]{@{}
  >{\raggedright\arraybackslash}p{(\linewidth - 6\tabcolsep) * \real{0.2558}}
  >{\raggedright\arraybackslash}p{(\linewidth - 6\tabcolsep) * \real{0.2558}}
  >{\raggedright\arraybackslash}p{(\linewidth - 6\tabcolsep) * \real{0.3256}}
  >{\raggedright\arraybackslash}p{(\linewidth - 6\tabcolsep) * \real{0.1628}}@{}}
\toprule\noalign{}
\begin{minipage}[b]{\linewidth}\raggedright
Milestone
\end{minipage} & \begin{minipage}[b]{\linewidth}\raggedright
EU target
\end{minipage} & \begin{minipage}[b]{\linewidth}\raggedright
Local target
\end{minipage} & \begin{minipage}[b]{\linewidth}\raggedright
Notes
\end{minipage} \\
\midrule\noalign{}
\endhead
\bottomrule\noalign{}
\endlastfoot
\textbf{Inventory baseline} & 2026‑03‑31 & 2026‑03‑15 & lock CBOM
scope \\
\textbf{Pilot migrations live} & 2026‑12‑31 & 2026‑11‑30 & include
telemetry \& rollback \\
\textbf{High‑risk systems PQC‑ready} & 2030‑12‑31 & 2030‑06‑30 & 6‑month
buffer for audit \\
\textbf{Medium‑risk systems PQC‑ready} & 2035‑12‑31 & 2035‑06‑30 &
contingent buffer \\
\end{longtable}

\begin{quote}
\emph{Rationale} -- buffer dates absorb supplier slippage, new standard
releases (FIPS~206, ISO/ETSI), or geopolitical disruptions.
\end{quote}

\subsubsection{10.5.2~~Work‑package design}\label{workpackage-design}

Break down migrations into \textbf{Epics} (e.g., \emph{TLS~Stack
Upgrade}) and \textbf{Stories} (e.g., \emph{enable hybrid Kyber in nginx
1.25}). Attach: * Definition of Done (test cases, security sign‑off) *
Estimated story points \& duration * Owner team and SME reviewers

\subsubsection{10.5.3~~Capacity \& cost
modelling}\label{capacity-cost-modelling}

\begin{itemize}
\tightlist
\item
  \textbf{FTE mapping} -- match story points to sprint velocity.
\item
  \textbf{External spend} -- licences, new HSMs, PKI vouchers, test‑bed
  cloud costs.
\item
  \textbf{Contingency reserve} -- 15\,\% of total CapEx based on Monte
  Carlo simulation of schedule risk.
\end{itemize}

\subsubsection{10.5.4~~Governance calendar}\label{governance-calendar}

Publish quarterly steering reviews and monthly working‑group
checkpoints. Each high‑risk migration has a \emph{go/no‑go gate} with
rollback cut‑off defined.

\subsection{10.6~~Step~3~~Readiness preparation
(ongoing)}\label{step-3-readiness-preparation-ongoing}

\subsubsection{10.6.1~~Supplier alignment}\label{supplier-alignment}

\begin{itemize}
\tightlist
\item
  Embed \emph{PQC‑ready clause} requiring CBOM + SPDX attestation with
  FIPS‑cert IDs by 2028.
\item
  Incentivise via payment milestones -- 10\,\% retainage until PQC
  compliance confirmed.
\end{itemize}

\subsubsection{10.6.2~~Toolchain hardening}\label{toolchain-hardening}

\begin{longtable}[]{@{}
  >{\raggedright\arraybackslash}p{(\linewidth - 2\tabcolsep) * \real{0.4286}}
  >{\raggedright\arraybackslash}p{(\linewidth - 2\tabcolsep) * \real{0.5714}}@{}}
\toprule\noalign{}
\begin{minipage}[b]{\linewidth}\raggedright
Tool category
\end{minipage} & \begin{minipage}[b]{\linewidth}\raggedright
Minimum capability
\end{minipage} \\
\midrule\noalign{}
\endhead
\bottomrule\noalign{}
\endlastfoot
\textbf{CI/CD scanner} & Detect lattice or hash‑based algorithm support,
block RSA‑2048 certs \\
\textbf{Traffic analyser} & Real‑time handshake cipher suite
telemetry \\
\textbf{HSM firmware} & Supports ML‑KEM‑768, ML‑DSA‑5, hybrid
wrapping \\
\end{longtable}

\subsubsection{10.6.3~~Skills uplift}\label{skills-uplift}

Deliver a role‑based training matrix: * \textbf{Dev‑ops} -- PQC
libraries, hybrid handshake patterns (2‑day workshop) * \textbf{IT Ops}
-- firmware‑signing, key rotation (1‑day lab) * \textbf{Risk officers}
-- QARS methodology, reporting dashboard (webinar + playbook)

\subsubsection{10.6.4~~Fallback planning}\label{fallback-planning}

Run \emph{table‑top exercises} for: 1. PQC handshake failure in
production causing 5xx spike. 2. Upstream library CVE requiring
emergency algorithm swap. 3. Supplier unable to deliver CBOM by
contractual date.

Each scenario results in a \emph{response run‑book} with RACI mapping
and MTTR target.

\subsection{10.7~~Outputs and
deliverables}\label{outputs-and-deliverables}

\begin{itemize}
\tightlist
\item
  \textbf{Master roadmap} (interactive Gantt or Kanban) stored in PMO
  repository.
\item
  \textbf{Budget \& resource plan} linked to finance system cost
  centres.
\item
  \textbf{Supplier tracker} -- contract ID, CBOM status, PQC clause
  compliance.
\item
  \textbf{Risk‑adjusted timeline} -- spreadsheet showing QARS, PI, and
  planned migration date per asset.
\end{itemize}

All deliverables should version via Git (for content) and
SharePoint/Confluence (for presentation decks). Use semantic version
tags (e.g., \texttt{roadmap‑v1.1.0}) to sync with Document Control.

\subsection{10.8~~Quality gates \& KPIs}\label{quality-gates-kpis}

\begin{longtable}[]{@{}
  >{\raggedright\arraybackslash}p{(\linewidth - 6\tabcolsep) * \real{0.1250}}
  >{\raggedright\arraybackslash}p{(\linewidth - 6\tabcolsep) * \real{0.3958}}
  >{\raggedright\arraybackslash}p{(\linewidth - 6\tabcolsep) * \real{0.2083}}
  >{\raggedright\arraybackslash}p{(\linewidth - 6\tabcolsep) * \real{0.2708}}@{}}
\toprule\noalign{}
\begin{minipage}[b]{\linewidth}\raggedright
Gate
\end{minipage} & \begin{minipage}[b]{\linewidth}\raggedright
Artefacts required
\end{minipage} & \begin{minipage}[b]{\linewidth}\raggedright
Approver
\end{minipage} & \begin{minipage}[b]{\linewidth}\raggedright
KPI trigger
\end{minipage} \\
\midrule\noalign{}
\endhead
\bottomrule\noalign{}
\endlastfoot
\textbf{G1 -- Inventory lock} & CBOM freeze, gap register & CISO &
\textless\,95\,\% asset coverage \\
\textbf{G2 -- Pilot go‑live} & Run‑book, rollback plan, test report &
Head of Ops & Error rate \textgreater\,0.1\,\% \\
\textbf{G3 -- High‑risk cut‑over} & External audit attestation &
Regulator liaison & Audit finding severity \textgreater\,``medium'' \\
\textbf{G4 -- Programme closure} & Lessons‑learned report, metrics
dashboard & Board & MTTR‑C \textgreater\,30\,days \\
\end{longtable}

Key performance indicators track \emph{Predictability} (variance
vs.~baseline), \emph{Quality} (defects, CVEs), and
\emph{Crypto‑compliance} (\% PQC certs).

\subsection{10.9~~Common pitfalls \& how to avoid
them}\label{common-pitfalls-how-to-avoid-them}

\begin{enumerate}
\def\labelenumi{\arabic{enumi}.}
\tightlist
\item
  \textbf{Over‑reliance on vendor roadmaps} -- mitigate by testing
  open‑source PQC libraries in parallel.
\item
  \textbf{Ignoring hidden dependencies} (e.g., SSO tokens signed with
  RSA) -- mandate \emph{dependency graph export} before migration.
\item
  \textbf{Resource starvation} during long tail of low‑risk systems --
  secure multi‑year budget with ring‑fenced FTEs.
\item
  \textbf{One‑shot big‑bang} migrations -- favour \emph{incremental
  hybrid} roll‑outs with fast rollback.
\item
  \textbf{Communication gaps} -- publish monthly progress dashboards to
  executives and teams.
\end{enumerate}

\subsection{10.10~~References}\label{references-2}

\begin{itemize}
\tightlist
\item
  European Union (2025). \emph{Coordinated Implementation Roadmap for
  Post‑Quantum Cryptography}.
\item
  TNO (2024). \emph{Post‑Quantum Cryptography Handbook}.
\item
  NIST (2024). \emph{FIPS~203,~204,~205}.
\item
  Mosca, M., et~al.~(2023). ``Cloud migration timelines for quantum
  risk''.
\end{itemize}

\section{11 E -- Execution \& Migration}\label{e-execution-migration-1}

\begin{quote}
\textbf{Purpose of this chapter} -- provide a practical playbook for
migrating systems from quantum‑vulnerable cryptography to post‑quantum
or hybrid primitives while maintaining service continuity, performance,
and compliance. It covers deployment models, testing strategies,
rollback procedures and quality gates.
\end{quote}

\subsection{11.1 Guiding principles
(≈\,120~words)}\label{guiding-principles-120-words}

\begin{enumerate}
\def\labelenumi{\arabic{enumi}.}
\tightlist
\item
  \textbf{Hybrid first} -- pair PQC primitives with existing RSA/ECDHE
  until ecosystem maturity allows full cut‑over.
\item
  \textbf{Incremental roll‑out} -- deploy to a small blast radius,
  monitor, then expand.
\item
  \textbf{Telemetry‑driven} -- measure handshake success, latency, error
  rates in real time.
\item
  \textbf{Reversible} -- every deployment must include an automated
  rollback path.
\item
  \textbf{Compliance aligned} -- FIPS‑approved parameter sets, algorithm
  policy enforced via crypto providers.
\end{enumerate}

\subsection{11.2 Migration patterns
(≈\,250~words)}\label{migration-patterns-250-words}

\begin{longtable}[]{@{}
  >{\raggedright\arraybackslash}p{(\linewidth - 8\tabcolsep) * \real{0.0769}}
  >{\raggedright\arraybackslash}p{(\linewidth - 8\tabcolsep) * \real{0.1615}}
  >{\raggedright\arraybackslash}p{(\linewidth - 8\tabcolsep) * \real{0.4462}}
  >{\raggedright\arraybackslash}p{(\linewidth - 8\tabcolsep) * \real{0.1385}}
  >{\raggedright\arraybackslash}p{(\linewidth - 8\tabcolsep) * \real{0.1769}}@{}}
\toprule\noalign{}
\begin{minipage}[b]{\linewidth}\raggedright
Pattern ID
\end{minipage} & \begin{minipage}[b]{\linewidth}\raggedright
Use case
\end{minipage} & \begin{minipage}[b]{\linewidth}\raggedright
Description
\end{minipage} & \begin{minipage}[b]{\linewidth}\raggedright
Pros
\end{minipage} & \begin{minipage}[b]{\linewidth}\raggedright
Cons
\end{minipage} \\
\midrule\noalign{}
\endhead
\bottomrule\noalign{}
\endlastfoot
\textbf{M‑H‑TLS} & Web/API TLS traffic & TLS~1.3 hybrid KEM: X25519 +
ML‑KEM‑768 (RFC~9399) & Minimal latency; browsers in test builds &
Larger ClientHello (\textasciitilde3\,kB) \\
\textbf{M‑H‑SSH} & Admin shell access & OpenSSH 9.4c with ECDH‑SHA2 +
Kyber768~+ dilithium keys & Easy CI integration & Requires updated
clients \\
\textbf{M‑PKI‑Nested} & Code signing & RSA‑2048 + Dilithium cert chain
(nested signatures) & Backwards compatible & 2× cert size \\
\textbf{M‑Hash‑FW} & IoT firmware updates & SPHINCS+ (128s) detached
signature; verify in bootloader & Stateless, audit‑friendly & 1\,MB
signature size \\
\textbf{M‑Sym‑Wrap} & Large data archives & AES‑256‑GCM data, key
wrapped with Kyber1024 then stored & Separates data-at-rest from PQC
cadence & Key management overhead \\
\end{longtable}

\subsection{11.3 Deployment workflow
(≈\,200~words)}\label{deployment-workflow-200-words}

\begin{enumerate}
\def\labelenumi{\arabic{enumi}.}
\tightlist
\item
  \textbf{Readiness checkpoint} -- ensure Stage~R work‑package passes
  go/no‑go gate.
\item
  \textbf{Pre‑prod lab} -- replicate production traffic with synthetic
  load; collect baseline metrics.
\item
  \textbf{Canary release} -- enable PQC for 1\,\% of traffic or a single
  AZ/node.
\item
  \textbf{Observation window} -- monitor KPIs (handshake
  success\,≥\,99.9\,\%, latency ≤\,+5\,ms).
\item
  \textbf{Gradual ramp‑up} -- double traffic every 24\,h if KPIs green.
\item
  \textbf{Full rollout} -- 100\,\% production traffic.
\item
  \textbf{Post‑deployment audit} -- verify cert chains, scan for
  deprecated algorithms.
\end{enumerate}

Automated scripts (\texttt{scripts/deploy/hybrid\_tls.sh}) orchestrate
feature flags via Envoy or nginx annotations.

\subsection{11.4 Testing strategy
(≈\,180~words)}\label{testing-strategy-180-words}

\begin{longtable}[]{@{}
  >{\raggedright\arraybackslash}p{(\linewidth - 4\tabcolsep) * \real{0.1959}}
  >{\raggedright\arraybackslash}p{(\linewidth - 4\tabcolsep) * \real{0.2680}}
  >{\raggedright\arraybackslash}p{(\linewidth - 4\tabcolsep) * \real{0.5361}}@{}}
\toprule\noalign{}
\begin{minipage}[b]{\linewidth}\raggedright
Test type
\end{minipage} & \begin{minipage}[b]{\linewidth}\raggedright
Tool / framework
\end{minipage} & \begin{minipage}[b]{\linewidth}\raggedright
Success criterion
\end{minipage} \\
\midrule\noalign{}
\endhead
\bottomrule\noalign{}
\endlastfoot
Unit tests & Google Test / Catch2 & PQC library returns expected
ciphertext length \\
Integration tests & Docker Compose stack & Service handshake completes
in \textless\,100\,ms \\
Fuzz testing & libFuzzer + AFL++ & No crashes after 24\,h fuzzing \\
Interop tests & OQS‑OpenSSL ↔ OQS‑nginx & 100\,\% pass across selected
cipher suites \\
Performance bench & wrk2, k6, vegeta & Throughput impact ≤\,5\,\% of
baseline \\
Chaos drills & Pumba / TC‑netem & Rollback trigger within 30\,s of error
spike \\
\end{longtable}

Continuous Integration pipelines in GitLab run these stages; results
export to SonarQube and Grafana.

\subsection{11.5 Rollback \& contingency
(≈\,120~words)}\label{rollback-contingency-120-words}

Every deployment artefact includes: - \textbf{Feature flag} to disable
PQC handshake at runtime. - \textbf{Blue/green} or \textbf{canary}
deployment environment. - \textbf{Backup certificates/keys} staged and
tested. - \textbf{Automated playbook}
\texttt{scripts/rollback/hybrid\_tls\_revert.sh}.

Triggers: - Error rate \textgreater\,0.5\,\% sustained for 5\,min. -
Latency increase \textgreater\,50\,ms for 5\,min. - Security incident
flagged by SOC.

\subsection{11.6 Telemetry \& metrics
(≈\,150~words)}\label{telemetry-metrics-150-words}

\begin{longtable}[]{@{}
  >{\raggedright\arraybackslash}p{(\linewidth - 6\tabcolsep) * \real{0.1475}}
  >{\raggedright\arraybackslash}p{(\linewidth - 6\tabcolsep) * \real{0.5902}}
  >{\raggedright\arraybackslash}p{(\linewidth - 6\tabcolsep) * \real{0.0984}}
  >{\raggedright\arraybackslash}p{(\linewidth - 6\tabcolsep) * \real{0.1639}}@{}}
\toprule\noalign{}
\begin{minipage}[b]{\linewidth}\raggedright
Metric ID
\end{minipage} & \begin{minipage}[b]{\linewidth}\raggedright
Description
\end{minipage} & \begin{minipage}[b]{\linewidth}\raggedright
Target
\end{minipage} & \begin{minipage}[b]{\linewidth}\raggedright
Collector
\end{minipage} \\
\midrule\noalign{}
\endhead
\bottomrule\noalign{}
\endlastfoot
\textbf{E‑1} & PQC handshake success rate & ≥\,99.9\,\% & Envoy stats \\
\textbf{E‑2} & Median handshake latency (ms) & ≤\,+5\,ms & Prometheus \\
\textbf{E‑3} & Error 5xx ratio during rollout & ≤\,0.1\,\% & Loki
logs \\
\textbf{E‑4} & Deprecated cipher usage (per hour) & 0 & Zeek \\
\end{longtable}

Dashboards live at \texttt{Grafana\ ›~PAREK\ ›~Execution}.

\subsection{11.7 Quality gate G3 -- Production readiness
(≈\,80~words)}\label{quality-gate-g3-production-readiness-80-words}

CISO (security), CIO (availability) and CFO (budget) sign off when: 1.
All tests pass, KPIs within thresholds. 2. Backout plan validated in
staging. 3. Supplier HSM firmware certs present. 4. Compliance evidence
(FIPS~cert numbers) attached to release ticket.

\subsection{11.8 Documentation deliverables
(≈\,100~words)}\label{documentation-deliverables-100-words}

\begin{itemize}
\tightlist
\item
  \textbf{Migration run‑book} -- step‑by‑step with screenshots/log
  snippets.
\item
  \textbf{Risk acceptance record} -- signed PDF by risk owner.
\item
  \textbf{Change record} -- ITIL ticket with links to pipeline run IDs.
\item
  \textbf{Post‑implementation review} -- lessons learned, metric
  screenshots.
\end{itemize}

Stored in \texttt{Confluence\ ›~PQC\ ›~Execution} with version tags
matching Git tags.

\subsection{11.9 Common pitfalls \& mitigations
(≈\,120~words)}\label{common-pitfalls-mitigations-120-words}

\begin{enumerate}
\def\labelenumi{\arabic{enumi}.}
\tightlist
\item
  \textbf{TLS library mismatch} -- pin exact OQS‑OpenSSL version; run
  interop tests.
\item
  \textbf{Certificate‑size blow‑up} -- enable TLS~1.3 compression
  extensions or nested certs.
\item
  \textbf{Log parser breakage} -- update regex patterns to parse new
  cipher suite names.
\item
  \textbf{HSM queue overflow} -- capacity test firmware before prod.
\item
  \textbf{Shadow RSA glue code} -- static‑link scanners in CI.
\end{enumerate}

\subsection{11.10 Future roadmap
(≈\,90~words)}\label{future-roadmap-90-words}

\begin{itemize}
\tightlist
\item
  \textbf{Full PQ‑only mode} once browser vendors ship Kyber in stable
  channels (target~2029).
\item
  \textbf{Algorithm agility APIs} (e.g., libpqcrypto~v2) to hot‑swap
  parameter sets.
\item
  \textbf{Quantum‑safe VPN and email encryption pilots}
  (Stage~E‑2026‑Q4).
\end{itemize}

\subsection{11.11 References}\label{references-3}

\begin{enumerate}
\def\labelenumi{\arabic{enumi}.}
\tightlist
\item
  IETF (2023). \emph{RFC~9399 -- Hybrid Key Exchange in TLS~1.3}.
\item
  Open Quantum Safe (2024). \emph{OQS‑OpenSSL 4.1}.
\item
  NIST (2024). \emph{FIPS~203~/~204~/~205}.
\item
  ENISA (2025). \emph{Post‑Quantum Migration Patterns}.
\end{enumerate}

\section{12 K -- Key‑Governance \& Continuous
Improvement}\label{k-keygovernance-continuous-improvement-1}

\begin{quote}
\textbf{Purpose of this chapter} -- define how EU‑based organisations
maintain crypto‑agility, monitor post‑quantum cryptography (PQC)
compliance, and sustain supplier accountability after initial migrations
are complete. Governance mechanisms align with \textbf{NIS~2},
\textbf{DORA}, \textbf{GDPR}, the forthcoming \textbf{EU Cyber
Resilience Act (CRA)}, and ENISA good‑practice guidelines.
\end{quote}

\subsection{12.1 Governance objectives
(≈\,100~words)}\label{governance-objectives-100-words}

\begin{enumerate}
\def\labelenumi{\arabic{enumi}.}
\tightlist
\item
  \textbf{Assurance} -- demonstrate to EU supervisory authorities (e.g.,
  NIS~Cooperation~Group, ECB‑SSM, EBA) that PQC controls remain
  effective.
\item
  \textbf{Transparency} -- provide board‑level and regulator‑level
  dashboards for cryptographic health.
\item
  \textbf{Continuous agility} -- support hot‑swaps to new PQC algorithms
  (e.g., ML‑KEM→ML‑KEM‑1024) without business disruption.
\item
  \textbf{Incident resilience} -- detect, triage and remediate crypto
  failures within predefined Mean Time to Remediate Crypto (MTTR‑C)
  targets.
\end{enumerate}

\subsection{12.2 Organisational structure
(≈\,150~words)}\label{organisational-structure-150-words}

\begin{longtable}[]{@{}
  >{\raggedright\arraybackslash}p{(\linewidth - 6\tabcolsep) * \real{0.2917}}
  >{\raggedright\arraybackslash}p{(\linewidth - 6\tabcolsep) * \real{0.0750}}
  >{\raggedright\arraybackslash}p{(\linewidth - 6\tabcolsep) * \real{0.5333}}
  >{\raggedright\arraybackslash}p{(\linewidth - 6\tabcolsep) * \real{0.1000}}@{}}
\toprule\noalign{}
\begin{minipage}[b]{\linewidth}\raggedright
Body
\end{minipage} & \begin{minipage}[b]{\linewidth}\raggedright
Frequency
\end{minipage} & \begin{minipage}[b]{\linewidth}\raggedright
Mandate
\end{minipage} & \begin{minipage}[b]{\linewidth}\raggedright
EU reference
\end{minipage} \\
\midrule\noalign{}
\endhead
\bottomrule\noalign{}
\endlastfoot
\textbf{PAREK Steering Committee} & Quarterly & Approve metrics, budget,
policy updates & NIS~2 Art.20 (management oversight) \\
\textbf{Crypto Governance Office (CGO)} & Monthly & Operate dashboards,
coordinate audits, own algorithm policy & ENISA Good Practice 4.2 \\
\textbf{Crypto Review Board (CRB)} & Ad hoc & Assess new
algorithms/parameters, sanction emergency swaps & ETSI TS~119~996
input \\
\textbf{Supplier Cryptography Board (SCB)} & Quarterly & Review Tier‑1
supplier attestation and CBOM status & CRA Art.35 (supplier
obligations) \\
\end{longtable}

Role mappings live in \texttt{part3/14-raci.md}.

\subsection{12.3 Policy stack
(≈\,120~words)}\label{policy-stack-120-words}

\begin{enumerate}
\def\labelenumi{\arabic{enumi}.}
\tightlist
\item
  \textbf{Cryptographic Policy (CP‑01)} -- lists approved algorithms,
  key lengths and protocols; revision every 6~months.
\item
  \textbf{Key Management Standard (KMS‑EU‑02)} -- describes lifecycle
  (generation, storage in EU Qualified Trust Service Provider (QTSP)
  HSMs, rotation, destruction).
\item
  \textbf{Algorithm Deprecation Procedure (ADP‑03)} -- triggers,
  timelines and communication templates for banning weak algorithms.
\item
  \textbf{Supplier Cryptography Policy (SCP‑04)} -- references §13 PQC
  Annex, aligns with CRA Article~10.
\end{enumerate}

Policies are version‑controlled in Git (\texttt{/policy/*}) and
published to the intranet Confluence space \texttt{PQC/Policies}.

\subsection{12.4 Metrics \& KPIs
(≈\,200~words)}\label{metrics-kpis-200-words}

\begin{longtable}[]{@{}
  >{\raggedright\arraybackslash}p{(\linewidth - 6\tabcolsep) * \real{0.0857}}
  >{\raggedright\arraybackslash}p{(\linewidth - 6\tabcolsep) * \real{0.6857}}
  >{\raggedright\arraybackslash}p{(\linewidth - 6\tabcolsep) * \real{0.0857}}
  >{\raggedright\arraybackslash}p{(\linewidth - 6\tabcolsep) * \real{0.1429}}@{}}
\toprule\noalign{}
\begin{minipage}[b]{\linewidth}\raggedright
KPI ID
\end{minipage} & \begin{minipage}[b]{\linewidth}\raggedright
Metric
\end{minipage} & \begin{minipage}[b]{\linewidth}\raggedright
Target
\end{minipage} & \begin{minipage}[b]{\linewidth}\raggedright
EU linkage
\end{minipage} \\
\midrule\noalign{}
\endhead
\bottomrule\noalign{}
\endlastfoot
\textbf{K‑1} & \% Assets with valid CBOM (\textless\,24\,h old) &
≥\,98\,\% & CRA Art.23 (SBOM/CBOM) \\
\textbf{K‑2} & Unsupported algorithm instances detected & 0 & NIS~2
Art.21 (technical measures) \\
\textbf{K‑3} & Mean Time to Remediate Crypto (MTTR‑C) & ≤\,30\,days
high‑risk; ≤\,90\,days medium & DORA RTS on ICT risk \\
\textbf{K‑4} & \% Supplier attestations received on time & ≥\,95\,\% &
CRA Art.35 \\
\textbf{K‑5} & Annual crypto penetration test coverage & 100\,\% Tier‑1,
80\,\% Tier‑2 & EBA Guidelines (ICT security) \\
\end{longtable}

All KPIs surface in Grafana dashboard
\texttt{PQC\ ›\ Governance\ ›\ EU\ Metrics} and feed quarterly NIS~2
reports.

\subsection{12.5 Continuous CBOM scanning
(≈\,150~words)}\label{continuous-cbom-scanning-150-words}

A \textbf{CBOM Delta Scanner} (Rust microservice) polls the CBOM graph
database hourly, compares it with the last approved baseline and flags:
- \textbf{Additions} -- new algorithms or keys not in policy. -
\textbf{Deletions} -- removed assets (possible shadow IT). -
\textbf{Parameter drift} -- changed key size or version.

Alerts integrate with ServiceNow (CIRF module). False positives must be
closed within 72\,h. All deltas export to
\texttt{assets/reports/cbom-delta-YYYY‑MM‑DD.csv} for audit evidence.

\subsection{12.6 Algorithm lifecycle management
(≈\,180~words)}\label{algorithm-lifecycle-management-180-words}

\subsubsection{12.6.1 Evaluation pipeline}\label{evaluation-pipeline}

\begin{enumerate}
\def\labelenumi{\arabic{enumi}.}
\tightlist
\item
  \textbf{Research~▶ Intake} -- CGO tracks NIST, ETSI, CEN/CENELEC
  outputs.
\item
  \textbf{Lab benchmark} -- CRB benchmarks latency, CPU, memory on
  reference workloads.
\item
  \textbf{Security review} -- external academic peer review (EU PQC
  Consortium).
\item
  \textbf{Pilot flag} -- enable new algorithm behind feature flag for
  selected services.
\item
  \textbf{Policy update} -- if successful, CP‑01 revision published.
\end{enumerate}

\subsubsection{12.6.2 Deprecation stages}\label{deprecation-stages}

\begin{longtable}[]{@{}lll@{}}
\toprule\noalign{}
Stage & Marker & Timeline \\
\midrule\noalign{}
\endhead
\bottomrule\noalign{}
\endlastfoot
\textbf{Proposed} & New candidate algorithm in ETSI draft & 0\,months \\
\textbf{Approved} & Added to CP‑01 & +6\,mths \\
\textbf{Mandatory} & Required for all new deployments & +18\,mths \\
\textbf{Forbidden} & Outgoing algorithm banned & +36\,mths \\
\end{longtable}

Communication packs sent via email and intranet; affected product owners
get Jira tasks auto‑generated.

\subsection{12.7 Incident response \& reporting
(≈\,150~words)}\label{incident-response-reporting-150-words}

Crypto incidents are handled under the \textbf{EU NIS~2 major incident}
framework:

\begin{enumerate}
\def\labelenumi{\arabic{enumi}.}
\tightlist
\item
  \textbf{Detection} -- SOC rule ``unsupported\_ciphersuite'' fires.
\item
  \textbf{Initial report} -- Incident Response Team logs case in
  TheHive; notif to national CSIRT within 24\,h.
\item
  \textbf{Containment} -- activate rollback script or key rotation.
\item
  \textbf{Eradication} -- remove bad certs, patch firmware.
\item
  \textbf{Post‑incident report} -- deliver ENISA‑template report within
  72\,h to competent authority.
\item
  \textbf{Lessons‑learned review} -- CRB updates ADP‑03 or KMS‑EU‑02.
\end{enumerate}

All steps timestamped; evidence archived in EU datacentre (GDPR
compliant).

\subsection{12.8 Audit \& assurance
(≈\,120~words)}\label{audit-assurance-120-words}

\begin{itemize}
\tightlist
\item
  \textbf{Internal audit} -- annual review aligned with ISAE~3402,
  reports to Audit Committee.
\item
  \textbf{External audit} -- Big~4 or qualified auditor validates KPIs,
  CBOM process, compliance with CRA and NIS~2.
\item
  \textbf{Regulator review} -- ECB‑SSM may request additional evidence
  for systemically important banks; DORA mandates ICT third‑party risk
  audits.
\end{itemize}

Audit findings tracked in Jira project \texttt{AUDIT‑PQC}; remediation
owned by CGO.

\subsection{12.9 Integration with other PAREK stages
(≈\,100~words)}\label{integration-with-other-parek-stages-100-words}

\begin{itemize}
\tightlist
\item
  \textbf{From Stage~E} -- deployment telemetry populates KPIs K‑1 to
  K‑3.
\item
  \textbf{To Stage~P} -- CBOM deltas feed new asset discovery.
\item
  \textbf{To Stage~R} -- maturity scores influence roadmap
  reprioritisation.
\item
  \textbf{With Stage~A} -- incident metrics adjust Exposure Surface
  Index for QARS re‑runs.
\end{itemize}

\subsection{12.10 Future EU developments
(≈\,80~words)}\label{future-eu-developments-80-words}

The \textbf{EU Digital Identity Wallet} regulation (eIDAS~2) will
require PQC‑capable Qualified Electronic Signatures (QES) by 2030. The
\textbf{EU AI Act} may impose additional controls for cryptographic
integrity in AI systems. Governance policy CP‑01 plans review cycles
aligned to these legislative sunsets.

\subsection{12.11 References}\label{references-4}

\begin{enumerate}
\def\labelenumi{\arabic{enumi}.}
\tightlist
\item
  ENISA~(2025). \emph{Good Practices for Crypto‑Agility and Post‑Quantum
  Preparedness}.
\item
  European Commission~(2023). \emph{NIS~2 Directive}.
\item
  European Parliament~(2025). \emph{Cyber Resilience Act~-- final text}.
\item
  ECB‑SSM~(2024). \emph{Cyber Resilience Oversight Expectations for
  FMIs}.
\item
  ETSI~(2024). \emph{TS~119~996~-- Algorithm Agility Guidance}.
\item
  CEN/CENELEC~(2025). \emph{PQC Standards Roadmap}.
\end{enumerate}

\section{13 Supply‑Chain Integration}\label{supplychain-integration}

\begin{quote}
\textbf{Purpose of this chapter} -- embed post‑quantum cryptography
(PQC) requirements into the entire supplier life‑cycle so that every
external component, cloud service and piece of hardware entering the
organisation's environment supports PAREK objectives and timelines.
\end{quote}

\subsection{13.1 Why supply‑chain matters in the quantum
era}\label{why-supplychain-matters-in-the-quantum-era}

Modern digital estates are a mosaic of proprietary SaaS APIs,
open‑source libraries, OEM devices and managed service providers.
Research by ENISA shows that 75\,\% of successful crypto‑deprecation
projects failed \textbf{not} because internal teams resisted change but
because third‑party dependencies lagged two to three years behind
security roadmaps. Quantum migration exacerbates this risk: a single
RSA‑signed software update from a vendor can re‑introduce vulnerable
primitives across thousands of endpoints. Therefore, PQC adoption is no
longer an internal programme but a \textbf{supply‑chain transformation
endeavour}. Section\,13 defines the contractual hooks, technical
artefacts (CBOM/SBOM), validation workflows and governance forums
required to make suppliers first‑class citizens in the PAREK lifecycle.

\subsection{13.2 Scope and definitions}\label{scope-and-definitions}

\textbf{Supplier} means any external legal entity that designs, builds,
sells or operates software, hardware or services running in, or
interfacing with, the organisation's production or pre‑production
environments. This includes SaaS providers, IaaS/PaaS cloud vendors, OEM
hardware suppliers, open‑source project maintainers (where code is
bundled), consultants and contract developers. \textbf{Supply‑chain
integration} spans four control layers: \emph{onboarding},
\emph{contracting}, \emph{operation}, and \emph{termination}. The
chapter applies to all suppliers whose deliverables contain or rely on
cryptographic functions, regardless of whether those functions are
explicitly exposed to the organisation (e.g., TLS) or hidden inside
firmware.

\subsection{13.3 Objectives}\label{objectives}

\begin{enumerate}
\def\labelenumi{\arabic{enumi}.}
\tightlist
\item
  \textbf{Cryptographic transparency} -- every supplier must furnish a
  machine‑readable Cryptography Bill of Materials (CBOM) aligned to
  CycloneDX\,v1.6.
\item
  \textbf{PQC readiness} -- high‑risk suppliers deliver PQC‑capable
  builds by 2028; medium‑risk by 2031.
\item
  \textbf{Continuous assurance} -- suppliers attest quarterly that no
  unsupported algorithms (RSA≤2048, ECC P‑256, SHA‑1) appear in
  deliverables.
\item
  \textbf{Incident response} -- suppliers notify the organisation within
  24~hours of any crypto‑related CVE with a CVSS score\,≥\,7.0.
\end{enumerate}

\subsection{13.4 Supplier segmentation
model}\label{supplier-segmentation-model}

The organisation classifies suppliers into \textbf{three tiers}:

\begin{longtable}[]{@{}
  >{\raggedright\arraybackslash}p{(\linewidth - 6\tabcolsep) * \real{0.1210}}
  >{\raggedright\arraybackslash}p{(\linewidth - 6\tabcolsep) * \real{0.4013}}
  >{\raggedright\arraybackslash}p{(\linewidth - 6\tabcolsep) * \real{0.2548}}
  >{\raggedright\arraybackslash}p{(\linewidth - 6\tabcolsep) * \real{0.2229}}@{}}
\toprule\noalign{}
\begin{minipage}[b]{\linewidth}\raggedright
Tier
\end{minipage} & \begin{minipage}[b]{\linewidth}\raggedright
Criteria
\end{minipage} & \begin{minipage}[b]{\linewidth}\raggedright
Examples
\end{minipage} & \begin{minipage}[b]{\linewidth}\raggedright
Governance cadence
\end{minipage} \\
\midrule\noalign{}
\endhead
\bottomrule\noalign{}
\endlastfoot
\textbf{1 -- Strategic} & Provides mission‑critical platforms or handles
classified data. & Core banking engine, national ID cloud. & Quarterly
steering; on‑site audits. \\
\textbf{2 -- Operational} & Supports key business processes but without
systemic impact. & CRM SaaS, managed network. & Semi‑annual review;
remote audit. \\
\textbf{3 -- Commodity} & Easily replaceable, low data sensitivity. &
Peripheral hardware, bulk email gateway. & Annual self‑assessment. \\
\end{longtable}

The tier determines the depth of CBOM detail, test evidence, and
contract clauses required. Tier~1 suppliers must present signed CBOMs,
PQC migration roadmaps and evidence of internal crypto‑agility testing.
Tier~3 suppliers may supply a simplified attestation if they leverage a
certified Tier~1 sub‑provider.

\subsection{13.5 Contractual
requirements}\label{contractual-requirements}

All new or renewed contracts \textbf{must} include a \emph{PQC Annex}
covering:

\begin{enumerate}
\def\labelenumi{\arabic{enumi}.}
\tightlist
\item
  \textbf{CBOM delivery schedule} -- initial CBOM within 30~days of
  contract signature; refreshed artefact with each major release or
  monthly for SaaS.
\item
  \textbf{PQC migration milestones} -- align with the organisation's
  roadmap (§10):

  \begin{itemize}
  \tightlist
  \item
    Kyber/Dilithium hybrid capability in test by \textbf{2027‑12‑31}.
  \item
    FIPS‑validated PQC primitives in production by \textbf{2030‑06‑30}
    for Tier\,1; \textbf{2031‑12‑31} for Tier\,2.
  \end{itemize}
\item
  \textbf{Algorithm deprecation clause} -- supplier shall not introduce
  or re‑enable algorithms listed on the organisation's \emph{Forbidden
  Algorithm List} (FAL).
\item
  \textbf{Crypto incident SLA} -- acknowledge within 2~business hours;
  provide root‑cause analysis within 5~working days.
\item
  \textbf{Audit \& testing rights} -- organisation may perform
  penetration tests focused on cryptographic endpoints once per calendar
  year, subject to 10~days' notice.
\item
  \textbf{Termination for non‑compliance} -- failure to meet milestone
  dates may trigger penalty fees up to 5\,\% of annual contract value or
  early termination.
\end{enumerate}

\begin{quote}
\emph{Tip~--} Legal teams should store the PQC Annex as a standalone
template (assets/contracts/pqc‑annex.docx) to streamline procurement.
All clauses reference external artefacts (CBOM spec, FAL) by version
number to avoid re‑negotiation when the lists update.
\end{quote}

\subsection{13.6 Technical artefacts and interfaces
(≈\,300~words)}\label{technical-artefacts-and-interfaces-300-words}

\subsubsection{13.6.1 Cryptography Bill of Materials
(CBOM)}\label{cryptography-bill-of-materials-cbom}

A CBOM is a JSON document (CycloneDX schema
\texttt{component:type="cryptography"}) listing:

\begin{itemize}
\tightlist
\item
  Algorithm (e.g., \texttt{rsa2048}, \texttt{ml‑kem‑768})
\item
  Protocol context (\texttt{tls1.3}, \texttt{ssh2}) and key sizes
\item
  Certificates or key IDs, including expiry and usage (signing,
  encryption)
\item
  Hardware anchoring (TPM, HSM model \& firmware version)
\item
  Compliance tags (FIPS~203, CC~EAL4+)
\end{itemize}

Suppliers must sign the CBOM using DSSE (in‑toto) and attach the
signature envelope as \texttt{*.cbom.sig}.

\subsubsection{13.6.2 SBOM‑CBOM linkage}\label{sbomcbom-linkage}

If a supplier already provides a Software Bill of Materials (SBOM), the
CBOM should reference SBOM components via \texttt{bom‑link} for
traceability. Example snippet:

\begin{Shaded}
\begin{Highlighting}[]
\FunctionTok{\{}
  \DataTypeTok{"bom‑link"}\FunctionTok{:} \StringTok{"urn:uuid:123e4567‑e89b‑12d3‑a456‑426614174000"}\FunctionTok{,}
  \DataTypeTok{"algorithm"}\FunctionTok{:} \StringTok{"ml‑kem‑768"}\FunctionTok{,}
  \DataTypeTok{"context"}\FunctionTok{:} \StringTok{"tls1.3"}\FunctionTok{,}
  \DataTypeTok{"status"}\FunctionTok{:} \StringTok{"hybrid"}
\FunctionTok{\}}
\end{Highlighting}
\end{Shaded}

\subsubsection{13.6.3 Delivery channels}\label{delivery-channels}

\begin{itemize}
\tightlist
\item
  \textbf{API} -- Tier~1 suppliers push CBOMs to \texttt{/api/v1/cbom}
  with OAuth~2.0 MTLS.
\item
  \textbf{S3 bucket} -- Tier~2 post JSON files to
  \texttt{s3://cbom‑uploads/\textless{}supplier\textgreater{}/\textless{}YYYY‑MM\textgreater{}/}.
\item
  \textbf{Email gateway} -- Tier~3 may email CBOMs signed with PGP;
  files routed to an ingest Lambda.
\end{itemize}

\subsubsection{13.6.4 Validation pipeline}\label{validation-pipeline}

Upon receipt, the organisation's \textbf{Crypto Intake Service}
performs:

\begin{enumerate}
\def\labelenumi{\arabic{enumi}.}
\tightlist
\item
  \textbf{Schema validation} -- rejects non‑conformant JSON.
\item
  \textbf{Signature check} -- DSSE verification against supplier's root
  cert.
\item
  \textbf{Policy scan} -- flag forbidden algorithms; raise ticket if
  found.
\item
  \textbf{Graph merge} -- append assets to central CBOM graph database.
\end{enumerate}

Failures trigger alerts to the \emph{Supplier Risk Queue} (Jira project
\texttt{SRQ}).

\subsection{13.7 Supplier assessment workflow
(≈\,180~words)}\label{supplier-assessment-workflow-180-words}

The following swim‑lane illustrates the annual assessment for a Tier~1
supplier:

\begin{verbatim}
Supplier ─┬─► Submit self‑assessment (questionnaire Q‑PQC‑001)
          │
Risk Team ├─► Score questionnaire (scale 0‑5)
          │
Crypto‑Sec COE ├─► Review CBOM → run lab tests (hybrid TLS)
              │
Procurement ─┴─► Evaluate penalties/bonuses → update contract
\end{verbatim}

Scores below 3 trigger a \textbf{Corrective Action Plan (CAP)}. CAP
tasks are tracked in the PAREK Programme backlog and must close within
90~days. Suppliers with sustained scores ≥4 across two consecutive
assessments may earn incentive rebates (1\,\% of contract value) or
preferred tender status.

\subsection{13.8 Tooling ecosystem
(≈\,150~words)}\label{tooling-ecosystem-150-words}

\begin{longtable}[]{@{}
  >{\raggedright\arraybackslash}p{(\linewidth - 4\tabcolsep) * \real{0.2923}}
  >{\raggedright\arraybackslash}p{(\linewidth - 4\tabcolsep) * \real{0.3538}}
  >{\raggedright\arraybackslash}p{(\linewidth - 4\tabcolsep) * \real{0.3538}}@{}}
\toprule\noalign{}
\begin{minipage}[b]{\linewidth}\raggedright
Function
\end{minipage} & \begin{minipage}[b]{\linewidth}\raggedright
Recommended tool / spec
\end{minipage} & \begin{minipage}[b]{\linewidth}\raggedright
Notes
\end{minipage} \\
\midrule\noalign{}
\endhead
\bottomrule\noalign{}
\endlastfoot
CBOM authoring & \texttt{cyclonedx‑python‑lib} & CLI + library
support \\
DSSE signing & \texttt{sigstore/cosign} & Leverage Fulcio CA \\
Validation pipeline & Custom Go microservice & Pluggable policy
engine \\
Graph storage & Neo4j or Amazon Neptune & Supports GraphQL API \\
Dashboard \& KPIs & Grafana + Prometheus & CBOM ingestion metrics \\
\end{longtable}

Integration playbooks live under \texttt{scripts/integration/} with
Terraform modules for AWS and Azure, ensuring suppliers can spin up the
same pipeline in their staging environments.

\subsection{13.9 Governance forums
(≈\,120~words)}\label{governance-forums-120-words}

\begin{itemize}
\tightlist
\item
  \textbf{Quarterly Supplier Cryptography Board (SCB)} -- chaired by the
  CISO; Tier~1 suppliers present migration progress. Outputs: meeting
  minutes, updated risk register.
\item
  \textbf{Monthly CBOM Ops Call} -- operational teams review ingestion
  metrics, false‑positive rates, upcoming schema changes.
\item
  \textbf{Annual PQC Summit} -- all suppliers invited; roadmap updates,
  lessons learned, and tooling demos shared. Attendance is a contract
  requirement for Tier\,1~and\,2 suppliers.
\end{itemize}

Governance artefacts are stored in SharePoint folder
\texttt{Governance/Supply‑Chain/} and referenced in Document Control.

\subsection{13.10 Integration with PAREK KPIs
(≈\,120~words)}\label{integration-with-parek-kpis-120-words}

The following metrics flow into §15:

\begin{longtable}[]{@{}
  >{\raggedright\arraybackslash}p{(\linewidth - 6\tabcolsep) * \real{0.1081}}
  >{\raggedright\arraybackslash}p{(\linewidth - 6\tabcolsep) * \real{0.4324}}
  >{\raggedright\arraybackslash}p{(\linewidth - 6\tabcolsep) * \real{0.2162}}
  >{\raggedright\arraybackslash}p{(\linewidth - 6\tabcolsep) * \real{0.2432}}@{}}
\toprule\noalign{}
\begin{minipage}[b]{\linewidth}\raggedright
KPI ID
\end{minipage} & \begin{minipage}[b]{\linewidth}\raggedright
Metric
\end{minipage} & \begin{minipage}[b]{\linewidth}\raggedright
Target
\end{minipage} & \begin{minipage}[b]{\linewidth}\raggedright
Data source
\end{minipage} \\
\midrule\noalign{}
\endhead
\bottomrule\noalign{}
\endlastfoot
\textbf{SC‑1} & \% suppliers with valid CBOM & ≥\,98\,\% & CBOM intake
logs \\
\textbf{SC‑2} & Mean CBOM ingestion latency & ≤\,2\,h & Pipeline
dashboard \\
\textbf{SC‑3} & \% Tier~1 PQC‑capable in test & 100\,\% by 2027‑Q4 &
Supplier roadmap \\
\textbf{SC‑4} & Crypto incident SLA breach count & 0 per quarter & GRC
ticket system \\
\end{longtable}

These KPIs feed the executive dashboard and are reported to regulators
under NIS‑2 critical‑infrastructure obligations.

\subsection{13.11 Common pitfalls \& mitigations
(≈\,120~words)}\label{common-pitfalls-mitigations-120-words-1}

\begin{enumerate}
\def\labelenumi{\arabic{enumi}.}
\tightlist
\item
  \textbf{Volume overwhelm} -- thousands of CBOM files per month;
  mitigate with batched digests and delta ingestion.
\item
  \textbf{Schema drift} -- suppliers using outdated CycloneDX versions;
  mandate schema~URI pinning and auto‑reject mismatches.
\item
  \textbf{Shadow suppliers} -- fourth‑party components hidden inside
  Tier~2 deliverables; enforce SBOM‑CBOM linkage and random audits.
\item
  \textbf{Legal bottlenecks} -- protracted clause negotiations; maintain
  pre‑approved PQC Annex template and fallback MSA language.
\item
  \textbf{False sense of security} -- signed CBOM ≠ secure crypto;
  supplement with periodic binary scans and penetration tests.
\end{enumerate}

\subsection{13.12 Future outlook
(≈\,90~words)}\label{future-outlook-90-words}

The EU Cyber Resilience Act may mandate \textbf{machine‑readable
vulnerability reporting} and real‑time disclosure notices. CycloneDX 2.0
will likely promote CBOM from \emph{extension} to \emph{first‑class
object}, adding richer lifecycle metadata (retirement, key‑rotation
schedules). Suppliers should budget time to adopt the new schema by
2027. Quantum‑safe HSM certifications (FIPS~203 level~3, CC~EAL5+) are
expected by 2026; contracts will update automatically when the
organisation's \emph{Approved Crypto Module List} refreshes.

\subsection{13.13 References}\label{references-5}

\begin{itemize}
\tightlist
\item
  CycloneDX~(2025). \emph{Cryptography Bill of Materials v1.6
  Specification}.
\item
  ENISA~(2024). \emph{Threat Landscape for Supply‑Chain Attacks}.
\item
  Sigstore~(2024). \emph{Cosign~2.0 -- Secure Artifact Signing}.
\item
  European Union~(2025). \emph{Cyber Resilience Act (final text)}.
\item
  TNO~(2024). \emph{Post‑Quantum Cryptography Handbook -- Supplier
  Section}.
\end{itemize}

\section{14 Roles, Responsibilities \&
RACI}\label{roles-responsibilities-raci}

\begin{quote}
\textbf{Purpose of this chapter} -- assign clear \textbf{R}esponsible,
\textbf{A}ccountable, \textbf{C}onsulted and \textbf{I}nformed ownership
for every stage, artefact and quality gate in the PAREK Framework, so
that decision‑making is transparent and compliant with EU governance
norms (NIS~2, DORA, GDPR, CRA).
\end{quote}

\subsection{14.1 RACI legend}\label{raci-legend}

\begin{longtable}[]{@{}
  >{\raggedright\arraybackslash}p{(\linewidth - 2\tabcolsep) * \real{0.0571}}
  >{\raggedright\arraybackslash}p{(\linewidth - 2\tabcolsep) * \real{0.9429}}@{}}
\toprule\noalign{}
\begin{minipage}[b]{\linewidth}\raggedright
Code
\end{minipage} & \begin{minipage}[b]{\linewidth}\raggedright
Meaning
\end{minipage} \\
\midrule\noalign{}
\endhead
\bottomrule\noalign{}
\endlastfoot
\textbf{R} & \textbf{Responsible} -- executes the task / delivers the
artefact \\
\textbf{A} & \textbf{Accountable} -- final sign‑off, owns success or
failure \\
\textbf{C} & \textbf{Consulted} -- provides input or subject‑matter
expertise \\
\textbf{I} & \textbf{Informed} -- kept up to date via dashboards,
reports or email \\
\end{longtable}

An individual or group may hold multiple codes but each task must have
\textbf{exactly one Accountable (A)}.

\subsection{14.2 Key organisational roles (EU
context)}\label{key-organisational-roles-eu-context}

\begin{longtable}[]{@{}
  >{\raggedright\arraybackslash}p{(\linewidth - 4\tabcolsep) * \real{0.0625}}
  >{\raggedright\arraybackslash}p{(\linewidth - 4\tabcolsep) * \real{0.4375}}
  >{\raggedright\arraybackslash}p{(\linewidth - 4\tabcolsep) * \real{0.5000}}@{}}
\toprule\noalign{}
\begin{minipage}[b]{\linewidth}\raggedright
Abbr.
\end{minipage} & \begin{minipage}[b]{\linewidth}\raggedright
Role / body
\end{minipage} & \begin{minipage}[b]{\linewidth}\raggedright
Typical EU alignment
\end{minipage} \\
\midrule\noalign{}
\endhead
\bottomrule\noalign{}
\endlastfoot
BoD & Board of Directors & NIS~2 Art.~20 -- management oversight \\
CISO & Chief Information Security Officer & NIS~2 Art.~21 -- technical
measures \\
CIO & Chief Information Officer & DORA ICT strategy \\
CFO & Chief Financial Officer & Budget approvals, risk cost modelling \\
CGO & Crypto Governance Office & Operates CP‑01, KMS‑EU‑02 policies \\
CRB & Crypto Review Board & Reviews new algorithms, deprecations \\
SCB & Supplier Cryptography Board & Oversees Tier‑1 suppliers (§13) \\
PMO & Programme Management Office & Tracks roadmap (§10) \\
Squad & Migration Squad (Dev‑Ops) & Executes Stage~E run‑books \\
Sup & Tier‑1 Supplier Representative & Provides CBOMs, attestations \\
\end{longtable}

\subsection{14.3 PAREK life‑cycle RACI
matrix}\label{parek-lifecycle-raci-matrix}

\begin{longtable}[]{@{}
  >{\raggedright\arraybackslash}p{(\linewidth - 20\tabcolsep) * \real{0.5775}}
  >{\centering\arraybackslash}p{(\linewidth - 20\tabcolsep) * \real{0.0423}}
  >{\centering\arraybackslash}p{(\linewidth - 20\tabcolsep) * \real{0.0423}}
  >{\centering\arraybackslash}p{(\linewidth - 20\tabcolsep) * \real{0.0423}}
  >{\centering\arraybackslash}p{(\linewidth - 20\tabcolsep) * \real{0.0423}}
  >{\centering\arraybackslash}p{(\linewidth - 20\tabcolsep) * \real{0.0423}}
  >{\centering\arraybackslash}p{(\linewidth - 20\tabcolsep) * \real{0.0423}}
  >{\centering\arraybackslash}p{(\linewidth - 20\tabcolsep) * \real{0.0423}}
  >{\centering\arraybackslash}p{(\linewidth - 20\tabcolsep) * \real{0.0423}}
  >{\centering\arraybackslash}p{(\linewidth - 20\tabcolsep) * \real{0.0423}}
  >{\centering\arraybackslash}p{(\linewidth - 20\tabcolsep) * \real{0.0423}}@{}}
\toprule\noalign{}
\begin{minipage}[b]{\linewidth}\raggedright
Stage / Artefact
\end{minipage} & \begin{minipage}[b]{\linewidth}\centering
BoD
\end{minipage} & \begin{minipage}[b]{\linewidth}\centering
CISO
\end{minipage} & \begin{minipage}[b]{\linewidth}\centering
CIO
\end{minipage} & \begin{minipage}[b]{\linewidth}\centering
CFO
\end{minipage} & \begin{minipage}[b]{\linewidth}\centering
CGO
\end{minipage} & \begin{minipage}[b]{\linewidth}\centering
CRB
\end{minipage} & \begin{minipage}[b]{\linewidth}\centering
SCB
\end{minipage} & \begin{minipage}[b]{\linewidth}\centering
PMO
\end{minipage} & \begin{minipage}[b]{\linewidth}\centering
Squad
\end{minipage} & \begin{minipage}[b]{\linewidth}\centering
Sup
\end{minipage} \\
\midrule\noalign{}
\endhead
\bottomrule\noalign{}
\endlastfoot
\textbf{P~-- Inventory} & I & A & C & I & R & C & I & I & R & R \\
--- CBOM schema \& tooling & I & C & C & I & A & C & I & I & R & R \\
--- Gap register & I & A & C & I & R & --- & I & C & R & C \\
\textbf{A~-- Quantum Risk Assessment} & I & A & C & C & R & C & I & C &
--- & C \\
--- QARS model weights & I & A & C & C & R & C & --- & C & --- & I \\
\textbf{R~-- Road‑map \& Readiness} & A & C & A & A & C & --- & C & R &
C & I \\
--- Budget baseline & A & C & C & A & C & --- & --- & R & --- & I \\
--- Supplier alignment plan & I & C & C & C & C & --- & A & R & --- &
R \\
\textbf{E~-- Execution \& Migration} & I & A & A & I & C & C & I & C & R
& R \\
--- Pilot roll‑out & I & C & A & I & C & C & I & C & R & R \\
--- Rollback execution & I & A & A & I & --- & --- & I & C & R & R \\
\textbf{K~-- Key Governance \& Improvement} & I & A & C & C & R & A & C
& I & C & C \\
--- KPI dashboard (K‑1 → K‑5) & I & A & C & C & R & C & C & I & C & I \\
--- Policy CP‑01 revision & I & C & C & I & A & R & I & I & --- & C \\
\textbf{Quality gates (G1‑G4)} & A & A & A & C & R & C & I & R & C &
I \\
\end{longtable}

Legend: \textbf{R = Responsible, A = Accountable, C = Consulted, I =
Informed}.

\subsection{14.4 Governance cadence (≈ 100
words)}\label{governance-cadence-100-words}

\begin{longtable}[]{@{}llll@{}}
\toprule\noalign{}
Forum & Frequency & Chair & Key outputs \\
\midrule\noalign{}
\endhead
\bottomrule\noalign{}
\endlastfoot
Steering Comm. & Quarterly & BoD & Budget, KPI review, escalations \\
CGO Weekly Ops & Weekly & CGO & CBOM delta report, KPI trend \\
CRB Algorithm & Ad~hoc & CRB & Algorithm approval/deprecation \\
SCB Supplier & Quarterly & CISO & Supplier compliance scorecard \\
\end{longtable}

Outputs feed Document Control and Stage~K dashboards.

\subsection{14.5 EU regulatory mapping (≈ 150
words)}\label{eu-regulatory-mapping-150-words}

\begin{itemize}
\tightlist
\item
  \textbf{NIS~2 Art.~20} -- Board is \emph{Accountable} for
  cybersecurity risk management → BoD holds \textbf{A} for quality
  gates.
\item
  \textbf{DORA (EU~2022/2554) Art.~12} -- ICT risk management → CIO
  shares \textbf{A} in Stages~R and~E.
\item
  \textbf{GDPR Art.~32} -- Security of processing → CISO ensures
  encryption strength (\textbf{A} in A, E, K).
\item
  \textbf{CRA Draft Art.~35} -- Supplier obligations → SCB assigns
  \textbf{A} to supplier compliance artefacts.
\end{itemize}

Alignment table stored under \texttt{assets/compliance/eu‑mapping.xlsx}.

\subsection{14.6 Role onboarding \& training (≈ 80
words)}\label{role-onboarding-training-80-words}

Each role receives a tailored induction pack (OneDrive folder
\texttt{Training/PAREK/\textless{}role\textgreater{}}), containing: -
Role charter \& RACI excerpt - Relevant policies (CP‑01, KMS‑EU‑02) -
Playbooks (incident response, algorithm review) - e‑Learning module
(SCORM) with EU regulatory quiz

Completion tracked via LMS; minimum pass score = 85\,\%.

\section{15 KPIs \& Reporting Dashboard''
authors}\label{kpis-reporting-dashboard-authors}

\begin{quote}
\textbf{Purpose of this chapter} -- define the key‑performance
indicators (KPIs), data flows and reporting dashboards that quantify
progress and operational health of the PAREK Programme. Metrics are
calibrated to EU supervisory expectations under \textbf{NIS~2},
\textbf{DORA} and the forthcoming \textbf{Cyber Resilience Act (CRA)}.
\end{quote}

\subsection{15.1 Why KPIs matter
(≈\,120~words)}\label{why-kpis-matter-120-words}

The EU regulatory shift from \emph{best‑effort} to \emph{demonstrable
assurance} means boards must produce hard evidence that quantum‑risk
controls are working. KPIs convert the qualitative objectives of PAREK
into quantifiable signals that:

\begin{enumerate}
\def\labelenumi{\arabic{enumi}.}
\tightlist
\item
  \textbf{Steer execution} -- highlight bottlenecks early.
\item
  \textbf{Inform regulators} -- feed mandatory NIS~2 incident and
  compliance reports.
\item
  \textbf{Drive supplier accountability} -- tie contract
  penalties/bonuses to measurable outcomes.
\end{enumerate}

Without robust KPIs, crypto‑agility devolves into one‑off migrations,
risking drift and audit findings.

\subsection{15.2 KPI taxonomy
(≈\,100~words)}\label{kpi-taxonomy-100-words}

KPIs are grouped into three tiers:

\begin{longtable}[]{@{}
  >{\raggedright\arraybackslash}p{(\linewidth - 6\tabcolsep) * \real{0.2381}}
  >{\raggedright\arraybackslash}p{(\linewidth - 6\tabcolsep) * \real{0.2024}}
  >{\raggedright\arraybackslash}p{(\linewidth - 6\tabcolsep) * \real{0.1071}}
  >{\raggedright\arraybackslash}p{(\linewidth - 6\tabcolsep) * \real{0.4524}}@{}}
\toprule\noalign{}
\begin{minipage}[b]{\linewidth}\raggedright
Tier
\end{minipage} & \begin{minipage}[b]{\linewidth}\raggedright
Audience
\end{minipage} & \begin{minipage}[b]{\linewidth}\raggedright
Frequency
\end{minipage} & \begin{minipage}[b]{\linewidth}\raggedright
Purpose
\end{minipage} \\
\midrule\noalign{}
\endhead
\bottomrule\noalign{}
\endlastfoot
\textbf{T1 -- Executive} & Board, regulators & Quarterly & Programme
health, compliance status \\
\textbf{T2 -- Operational} & CISO, CGO, PMO & Monthly & Stage‑level
performance, SLA breaches \\
\textbf{T3 -- Tactical} & Dev‑Ops squads & Daily & Deployment metrics,
incident telemetry \\
\end{longtable}

This chapter lists core Tier~1 and Tier~2 KPIs. Tactical metrics are
documented in Stage~E run‑books.

\subsection{15.3 Core KPI catalogue
(≈\,250~words)}\label{core-kpi-catalogue-250-words}

\begin{longtable}[]{@{}
  >{\raggedright\arraybackslash}p{(\linewidth - 12\tabcolsep) * \real{0.0488}}
  >{\raggedright\arraybackslash}p{(\linewidth - 12\tabcolsep) * \real{0.0610}}
  >{\raggedright\arraybackslash}p{(\linewidth - 12\tabcolsep) * \real{0.3049}}
  >{\raggedright\arraybackslash}p{(\linewidth - 12\tabcolsep) * \real{0.2683}}
  >{\raggedright\arraybackslash}p{(\linewidth - 12\tabcolsep) * \real{0.0854}}
  >{\raggedright\arraybackslash}p{(\linewidth - 12\tabcolsep) * \real{0.1220}}
  >{\raggedright\arraybackslash}p{(\linewidth - 12\tabcolsep) * \real{0.1098}}@{}}
\toprule\noalign{}
\begin{minipage}[b]{\linewidth}\raggedright
KPI ID
\end{minipage} & \begin{minipage}[b]{\linewidth}\raggedright
Stage link
\end{minipage} & \begin{minipage}[b]{\linewidth}\raggedright
Metric (EU aligned)
\end{minipage} & \begin{minipage}[b]{\linewidth}\raggedright
Calculation / data source
\end{minipage} & \begin{minipage}[b]{\linewidth}\raggedright
Target
\end{minipage} & \begin{minipage}[b]{\linewidth}\raggedright
Alert threshold
\end{minipage} & \begin{minipage}[b]{\linewidth}\raggedright
Reg. mapping
\end{minipage} \\
\midrule\noalign{}
\endhead
\bottomrule\noalign{}
\endlastfoot
\textbf{P‑1} & P & CBOM coverage &
\texttt{\#assets\ with\ valid\ CBOM\ /\ \#assets\ in\ scope} & ≥\,98\,\%
& \textless\,95\,\% & CRA~Art\,23 \\
\textbf{A‑1} & A & High‑risk assets (QARS~≥\,0.65) remaining & Count
from \texttt{risk\_registry.csv} & →\,0 by 2029‑Q4 &
\textgreater\,Baseline trendline & NIS~2 Art\,21 \\
\textbf{R‑1} & R & Schedule variance &
\texttt{planned\ finish\ –\ actual} (days) & ±\,0--10\,days &
\textgreater\,15\,days & DORA Art\,12 \\
\textbf{E‑1} & E & PQC handshake success rate & Envoy / Prometheus
metric & ≥\,99.9\,\% & \textless\,99.5\,\% & NIS\,CSIRT guidance \\
\textbf{E‑2} & E & Median handshake latency delta &
\texttt{PQC\_latency\ –\ baseline\_latency} & ≤\,+5\,ms &
\textgreater\,+10\,ms & ENISA perf. rec. \\
\textbf{K‑1} & K & Unsupported algorithm instances detected &
Zeek/Suricata rules & 0 & \textgreater\,0 & GDPR Art\,32 \\
\textbf{K‑2} & K & Mean Time to Remediate Crypto (MTTR‑C) -- high‑risk &
\texttt{Σ\ resolution\ time\ /\ \#\ incidents} & ≤\,30\,days &
\textgreater\,45\,days & DORA RTS \\
\textbf{SC‑1} & §13 & Suppliers with on‑time CBOM and attestation &
\texttt{\#compliant\ /\ \#total\ suppliers} & ≥\,95\,\% &
\textless\,90\,\% & CRA~Art\,35 \\
\textbf{M‑1} & §10 & Programme budget adherence &
\texttt{actual\ spend\ /\ budget} & ≤\,110\,\% & \textgreater\,120\,\% &
Board policy \\
\end{longtable}

\subsection{15.4 Data architecture
(≈\,150~words)}\label{data-architecture-150-words}

\begin{verbatim}
CBOM API ─┐                 ┌──► Postgres (risk_registry) ──► Grafana API
          │   Lambda ETL   │
Zeek logs ─┤──► Kafka bus ──┤
          │                 │
Jira API ──┘                 └──► Prometheus (E‑metrics) ───► Grafana API
\end{verbatim}

\begin{itemize}
\tightlist
\item
  \textbf{Ingest layer} -- Kafka collects CBOM deltas, Zeek alerts,
  deployment metrics.
\item
  \textbf{Storage layer} -- time‑series in Prometheus; relational in
  Postgres.
\item
  \textbf{Analytics layer} -- Python notebook
  (\texttt{scripts/analysis/kpi\_report.ipynb}) calculates monthly
  aggregates.
\item
  \textbf{Visual layer} -- Grafana dashboards; snapshots auto‑export as
  PNG for board packs.
\end{itemize}

All components run in EU datacentres (GDPR Art\,44 compliant). Access
controls via Azure AD groups.

\subsection{15.5 Dashboard design
(≈\,120~words)}\label{dashboard-design-120-words}

\subsubsection{Executive dashboard}\label{executive-dashboard}

\begin{itemize}
\tightlist
\item
  \textbf{Gauge} -- CBOM coverage (P‑1)
\item
  \textbf{Stacked bar} -- High/medium/low assets over time (A‑1)
\item
  \textbf{Line} -- Budget vs.~actual (M‑1)
\item
  \textbf{Heat‑map} -- Supplier compliance (SC‑1)
\end{itemize}

\subsubsection{Operational dashboard}\label{operational-dashboard}

\begin{itemize}
\tightlist
\item
  \textbf{Table} -- Unsupported algorithm findings (K‑1) by business
  unit
\item
  \textbf{Histogram} -- MTTR‑C distribution (K‑2)
\item
  \textbf{Sankey} -- Incident cause → resolution path
\item
  \textbf{Alert panel} -- Live E‑metrics (E‑1, E‑2)
\end{itemize}

Grafana JSON imports stored under
\texttt{assets/grafana/kpi\_dashboards/}.

\subsection{15.6 Governance \& review cadence
(≈\,120~words)}\label{governance-review-cadence-120-words}

\begin{longtable}[]{@{}
  >{\raggedright\arraybackslash}p{(\linewidth - 8\tabcolsep) * \real{0.3226}}
  >{\raggedright\arraybackslash}p{(\linewidth - 8\tabcolsep) * \real{0.1613}}
  >{\raggedright\arraybackslash}p{(\linewidth - 8\tabcolsep) * \real{0.1452}}
  >{\raggedright\arraybackslash}p{(\linewidth - 8\tabcolsep) * \real{0.2903}}
  >{\raggedright\arraybackslash}p{(\linewidth - 8\tabcolsep) * \real{0.0806}}@{}}
\toprule\noalign{}
\begin{minipage}[b]{\linewidth}\raggedright
Report
\end{minipage} & \begin{minipage}[b]{\linewidth}\raggedright
Audience
\end{minipage} & \begin{minipage}[b]{\linewidth}\raggedright
Frequency
\end{minipage} & \begin{minipage}[b]{\linewidth}\raggedright
Delivery channel
\end{minipage} & \begin{minipage}[b]{\linewidth}\raggedright
Owner
\end{minipage} \\
\midrule\noalign{}
\endhead
\bottomrule\noalign{}
\endlastfoot
KPI snapshot (PDF) & Board & Quarterly & SharePoint / email & PMO \\
KPI drill‑down deck & CGO & Monthly & Confluence & CGO \\
KPI raw export (CSV) & Regulators & Annual & SFTP to CSIRT & CISO \\
\end{longtable}

Each quarter, the Steering Committee reviews trend deltas. Any KPI
breaching alert threshold triggers a \textbf{Corrective Action Plan
(CAP)} logged in Jira.

\subsection{15.7 Continuous improvement loop
(≈\,100\,words)}\label{continuous-improvement-loop-100-words}

\begin{enumerate}
\def\labelenumi{\arabic{enumi}.}
\tightlist
\item
  \textbf{Detect} -- KPI alert fires.
\item
  \textbf{Diagnose} -- Root‑cause analysis meeting within 5~days.
\item
  \textbf{Decide} -- CRB or CGO selects remediation (e.g., policy tweak,
  supplier escalation).
\item
  \textbf{Deliver} -- Squad implements; KPI flagged ``watch'' for
  30~days.
\item
  \textbf{Document} -- Lessons‑learned stored in Confluence.
\end{enumerate}

KPIs themselves undergo annual review (Stage\,K). Weightings or new
metrics added through change‑control procedure CP‑01‑KPI‑UPDATE.

\subsection{15.8 EU regulatory reporting alignment
(≈\,120~words)}\label{eu-regulatory-reporting-alignment-120-words}

\begin{itemize}
\tightlist
\item
  \textbf{NIS~2} -- P‑1, K‑1, K‑2 feed into the \emph{Security Measures}
  section of the annual NIS~2 compliance report sent to the national
  CSIRT.
\item
  \textbf{DORA} -- K‑2, E‑metrics underpin the ICT~Risk Management
  template required by ESAs.
\item
  \textbf{CRA} -- SC‑1 and CBOM coverage support product security
  declarations.
\item
  \textbf{ECB TIBER‑EU} -- A‑1 trend informs the threat‑intelligence
  baseline for red‑team tests.
\end{itemize}

Mapping table maintained at \texttt{assets/compliance/kpi‑eu‑map.xlsx}.

\subsection{15.9 Common pitfalls \& mitigations
(≈\,100~words)}\label{common-pitfalls-mitigations-100-words-1}

\begin{enumerate}
\def\labelenumi{\arabic{enumi}.}
\tightlist
\item
  \textbf{Metric overload} -- focus on \textless\,15 KPIs; archive
  vanity metrics.
\item
  \textbf{Gaming the numbers} -- random audits of data sources;
  automated anomaly detection.
\item
  \textbf{Stale dashboards} -- CI job fails; alert on last data‑refresh
  timestamp \textgreater\,1~day.
\item
  \textbf{One‑size targets} -- calibrate KPIs per business unit; avoid
  blanket thresholds.
\end{enumerate}

\subsection{15.10 Next steps}\label{next-steps-2}

\begin{itemize}
\tightlist
\item
  Finalise Grafana dashboard JSON after Sections\,8--12 baseline
  metrics.
\item
  Include KPI snapshot in next Board pack (Q3~2025).
\item
  Schedule ENISA‑style KPI workshop for suppliers (Q4~2025).
\end{itemize}

\subsection{15.11 References}\label{references-6}

\begin{enumerate}
\def\labelenumi{\arabic{enumi}.}
\tightlist
\item
  ENISA (2024). \emph{Guidelines on KPIs for Cybersecurity Measures}.
\item
  European Commission (2023). \emph{NIS~2 Directive}.
\item
  European Parliament (2025). \emph{Cyber Resilience Act -- Final Text}.
\item
  EBA (2024). \emph{ICT Risk Management under DORA}.
\item
  Grafana Labs (2025). \emph{Best Practices for KPI Dashboards}.
\end{enumerate}

\section{16 Reference Architectures \&
Tooling}\label{reference-architectures-tooling}

\begin{quote}
\textbf{Purpose of this chapter} -- provide opinionated, EU‑aligned
reference architectures that engineering teams can adopt or adapt when
implementing PAREK migrations. Each pattern embraces open‑source
baselines, indicates where commercial substitutes may slot in, and
highlights regulatory hooks (NIS\,2, DORA, CRA, eIDAS~2).
\end{quote}

\subsection{16.1 Reading guide
(≈\,80\,words)}\label{reading-guide-80-words}

Each subsection presents:

\begin{enumerate}
\def\labelenumi{\arabic{enumi}.}
\tightlist
\item
  \textbf{Context} -- why the pattern matters.
\item
  \textbf{Diagram} -- ASCII or UML sketch.
\item
  \textbf{Component list} -- open‑source baseline + commercial
  alternatives.
\item
  \textbf{EU compliance notes} -- which articles/standards the pattern
  satisfies.
\item
  \textbf{Implementation tips} -- common pitfalls, performance notes.
\end{enumerate}

Full Terraform or Helm charts live in \texttt{assets/infra/}.

\subsection{\texorpdfstring{16.2 PQ‑ready PKI
(\emph{Pattern~RA‑PKI‑EU})}{16.2 PQ‑ready PKI (Pattern~RA‑PKI‑EU)}}\label{pqready-pki-pattern-rapkieu}

\subsubsection{16.2.1 Context}\label{context}

Most TLS, code‑signing and device‑auth chains depend on a X.509
hierarchy. Upgrading to hybrid or PQ‑only certificates without forklift
replacements requires a crypto‑agile PKI.

\subsubsection{16.2.2 Diagram}\label{diagram}

\begin{verbatim}
               EU Trust List (EUTL)
                      │
        ┌─────────────┴─────────────┐
        │        OFFLINE ROOT       │  (RSA‑4096 + Dilithium5)
        └──────┬──────────┬─────────┘
               │          │
        ┌──────┴───┐  ┌───┴────┐
        │  ISSUING │  │ ISSUING│   (ML‑DSA‑2, RSA‑4096)
        │   CA‑A   │  │  CA‑B  │
        └──┬───┬───┘  └───┬───┬─┘
           │   │         │   │
       ┌───┴─┐ └─┐   ┌───┴─┐ └─┐
       │ TLS │  IoT  │Code │  VPN  (leafs: hybrid certs)
\end{verbatim}

\subsubsection{16.2.3 Components}\label{components}

\begin{longtable}[]{@{}
  >{\raggedright\arraybackslash}p{(\linewidth - 4\tabcolsep) * \real{0.1270}}
  >{\raggedright\arraybackslash}p{(\linewidth - 4\tabcolsep) * \real{0.3651}}
  >{\raggedright\arraybackslash}p{(\linewidth - 4\tabcolsep) * \real{0.5079}}@{}}
\toprule\noalign{}
\begin{minipage}[b]{\linewidth}\raggedright
Function
\end{minipage} & \begin{minipage}[b]{\linewidth}\raggedright
Open‑source
\end{minipage} & \begin{minipage}[b]{\linewidth}\raggedright
Commercial
\end{minipage} \\
\midrule\noalign{}
\endhead
\bottomrule\noalign{}
\endlastfoot
CA core & EJBCA CE & Entrust PKIaaS EU \\
HSM & SoftHSM + PKCS\#11 & Thales Luna HSM7 (EU datacentre) \\
ACME & \texttt{certbot} (OQS‑patched) & Sectigo Certificate Manager \\
\end{longtable}

\subsubsection{16.2.4 EU compliance}\label{eu-compliance}

\begin{itemize}
\tightlist
\item
  \textbf{eIDAS~2} QES requirements → Offline root must be hosted in EU
  + QTSP audit.
\item
  \textbf{NIS\,2 Art.\,21} technical controls → dual control on root key
  ceremonies.
\end{itemize}

\subsubsection{16.2.5 Implementation tips}\label{implementation-tips}

\begin{itemize}
\tightlist
\item
  Use \textbf{nested signatures}: RSA‑4096 outer, Dilithium5 inner to
  satisfy legacy clients.
\item
  Test OCSP responders for 4\,kB cert sizes.
\end{itemize}

\subsection{\texorpdfstring{16.3 Hybrid TLS termination
(\emph{RA‑TLS‑HYB})}{16.3 Hybrid TLS termination (RA‑TLS‑HYB)}}\label{hybrid-tls-termination-ratlshyb}

\subsubsection{Context}\label{context-1}

Web/API gateways must negotiate Kyber + X25519 KEM yet preserve
performance.

\subsubsection{Diagram}\label{diagram-1}

\begin{verbatim}
Users ──► Cloudflare Zaraz (TLS 1.3) ─► Envoy Edge ─► Service Mesh ─► App Pods
            │ Kyber768+X25519           │ Kyber768+X25519     │ mTLS (OQS‑gRPC)
\end{verbatim}

\subsubsection{Components}\label{components-1}

\begin{longtable}[]{@{}lll@{}}
\toprule\noalign{}
Layer & OSS baseline & Commercial EU‑hosted \\
\midrule\noalign{}
\endhead
\bottomrule\noalign{}
\endlastfoot
CDN & Cloudflare beta PQC edge & Akamai Secure Edge PQC \\
Proxy & Envoy 1.31 + OQS‑BoringSSL & NGINX~Plus FIPS‑PQC module \\
mTLS & OQS‑gRPC & Istio with Thales DataShield \\
\end{longtable}

\subsubsection{EU notes}\label{eu-notes}

\begin{itemize}
\tightlist
\item
  CRA requires ``state‑of‑the‑art'' crypto → hybrid by 2026 meets
  ``state‑of‑the‑art'' definition.
\item
  GDPR Art.\,32 encryption → document cipher suite in RoPA.
\end{itemize}

\subsubsection{Tips}\label{tips}

\begin{itemize}
\tightlist
\item
  Enable \textbf{GREASE} support to avoid middlebox drops.
\item
  Capture baseline latency; expect +2‑5\,ms at 1\,kB handshake growth.
\end{itemize}

\subsection{\texorpdfstring{16.4 Secure code‑signing pipeline
(\emph{RA‑CODE‑SIGN})}{16.4 Secure code‑signing pipeline (RA‑CODE‑SIGN)}}\label{secure-codesigning-pipeline-racodesign}

\subsubsection{Diagram}\label{diagram-2}

\begin{verbatim}
Git Commit ─► CI Build ─► Cosign Sign (Dilithium2) ─► Rekor Transparency Log ─► Artifactory
\end{verbatim}

\subsubsection{Components}\label{components-2}

\begin{longtable}[]{@{}
  >{\raggedright\arraybackslash}p{(\linewidth - 4\tabcolsep) * \real{0.0909}}
  >{\raggedright\arraybackslash}p{(\linewidth - 4\tabcolsep) * \real{0.5606}}
  >{\raggedright\arraybackslash}p{(\linewidth - 4\tabcolsep) * \real{0.3485}}@{}}
\toprule\noalign{}
\begin{minipage}[b]{\linewidth}\raggedright
Step
\end{minipage} & \begin{minipage}[b]{\linewidth}\raggedright
Tool (OSS)
\end{minipage} & \begin{minipage}[b]{\linewidth}\raggedright
Alt (Commercial)
\end{minipage} \\
\midrule\noalign{}
\endhead
\bottomrule\noalign{}
\endlastfoot
Sign & \texttt{sigstore/cosign\ -\/-key\ dilithium.key} & Venafi
CodeSign Protect \\
Log & \texttt{sigstore/rekor} EU cluster & Ledger EU Notary \\
Verify & \texttt{cosign\ verify\ -\/-key\ dilithium.pub} & Jenkins PQC
plugin \\
\end{longtable}

\subsubsection{EU alignment}\label{eu-alignment}

\begin{itemize}
\tightlist
\item
  CRA mandates SBOM/CVEs disclosure → attach CBOM + SBOM via Sigstore
  DSSE.
\item
  DORA ICT ‑ data integrity → use Transparency Log proofs.
\end{itemize}

\subsection{\texorpdfstring{16.5 CBOM ingestion \& graph
(\emph{RA‑CBOM‑EU})}{16.5 CBOM ingestion \& graph (RA‑CBOM‑EU)}}\label{cbom-ingestion-graph-racbomeu}

\subsubsection{Diagram}\label{diagram-3}

\begin{verbatim}
Supplier CBOM JSON ─► API Gateway (OAuth) ─► Kafka topic `cbom.raw`
        │                                   │
        └───────── error queue ─────────────┘
                   │
             ETL (Rust Lambda) ─► Neo4j GraphDB ─► Grafana Dash
\end{verbatim}

\subsubsection{Components}\label{components-3}

\begin{longtable}[]{@{}lll@{}}
\toprule\noalign{}
Function & OSS & Commercial \\
\midrule\noalign{}
\endhead
\bottomrule\noalign{}
\endlastfoot
Gateway & Kong Gateway & Azure APIM EU \\
Queue & Kafka & AWS MSK (eu‑west‑1) \\
Graph & Neo4j Community & Amazon Neptune \\
\end{longtable}

\subsubsection{EU notes}\label{eu-notes-1}

\begin{itemize}
\tightlist
\item
  Store all supplier data within EEA to satisfy GDPR Art.\,44.
\end{itemize}

\subsection{\texorpdfstring{16.6 Crypto‑agile secret management
(\emph{RA‑SECRETS})}{16.6 Crypto‑agile secret management (RA‑SECRETS)}}\label{cryptoagile-secret-management-rasecrets}

\subsubsection{Diagram}\label{diagram-4}

\begin{verbatim}
Apps ► HashiCorp Vault (Transit Engine PQC plugin) ► HSM partition (ML‑KEM)
\end{verbatim}

\subsubsection{Implementation tips}\label{implementation-tips-1}

\begin{itemize}
\tightlist
\item
  Use \textbf{Key Versioning} to rotate to future algorithms (e.g.,
  ML‑KEM‑1024).
\item
  Enable \textbf{Key Type Tags} to block RSA key generation post‑2030.
\end{itemize}

\subsection{16.7 Mapping architectures to PAREK
stages}\label{mapping-architectures-to-parek-stages}

\begin{longtable}[]{@{}lll@{}}
\toprule\noalign{}
Stage & Primary reference pattern & Artefacts produced \\
\midrule\noalign{}
\endhead
\bottomrule\noalign{}
\endlastfoot
\textbf{P} & RA‑CBOM‑EU & CBOM graph export \\
\textbf{A} & (N/A) -- consumes CBOM & Risk registry \\
\textbf{R} & Integration of RA‑PKI‑EU & Roadmap epics \\
\textbf{E} & RA‑TLS‑HYB, RA‑CODE‑SIGN & Run‑books, metrics \\
\textbf{K} & RA‑SECRETS, monitoring stack & KPI dashboards \\
\end{longtable}

\subsection{16.8 EU compliance cross‑reference
(summary)}\label{eu-compliance-crossreference-summary}

\begin{longtable}[]{@{}llllll@{}}
\toprule\noalign{}
Pattern & NIS~2 & DORA & CRA & eIDAS~2 & GDPR \\
\midrule\noalign{}
\endhead
\bottomrule\noalign{}
\endlastfoot
RA‑PKI‑EU & ✔ & --- & --- & ✔ & --- \\
RA‑TLS‑HYB & ✔ & ✔ & ✔ & --- & ✔ \\
RA‑CODE‑SIGN & ✔ & ✔ & ✔ & ✔ & --- \\
RA‑CBOM‑EU & ✔ & ✔ & ✔ & --- & ✔ \\
RA‑SECRETS & ✔ & ✔ & --- & --- & ✔ \\
\end{longtable}

Full mapping sheet lives in
\texttt{assets/compliance/patterns‑eu‑matrix.xlsx}.

\subsection{16.9 Next steps}\label{next-steps-3}

\begin{itemize}
\tightlist
\item
  Pilot \textbf{RA‑TLS‑HYB} on staging APIs (Q4~2025).
\item
  Migrate code‑signing pipeline to Dilithium2 by Q1~2026.
\item
  Integrate CBOM graph with risk dashboard (Stage\,K KPI P‑1) before
  next NIS~2 audit.
\end{itemize}

\subsection{16.10 References}\label{references-7}

\begin{enumerate}
\def\labelenumi{\arabic{enumi}.}
\tightlist
\item
  ETSI (2024). \emph{TS~119~996 -- Algorithm Agility Principles}.
\item
  Sigstore (2025). \emph{PQC Roadmap}.
\item
  ENISA (2025). \emph{Architecture Patterns for Post‑Quantum Migration}.
\item
  Cloudflare (2025). \emph{Hybrid KEM Performance Whitepaper}.
\item
  CEN/CENELEC (2025). \emph{Guideline on PQC‑Ready PKIs}.
\end{enumerate}

\section{17 Glossary \& Acronyms (CEN/CENELEC \&
ISO‑aligned)}\label{glossary-acronyms-cencenelec-isoaligned}

\begin{quote}
\textbf{Purpose} -- harmonise terminology across the handbook and
supplier communications. Definitions derive, where possible, from
authoritative European standards: \textbf{CEN/CENELEC Guide~30:2015}
(\emph{European~Standardisation -- Vocabulary}), \textbf{EN~ISO/IEC
2382} (\emph{Information technology~-- Vocabulary}) and
\textbf{ETSI~TR~103~684} (\emph{Algorithm Agility and Post‑Quantum
Cryptography}). Where no official wording exists, the editorial team
supplies a consensual definition.
\end{quote}

\emph{Abbreviations are ordered alphabetically; initialisms are
uppercase, terms are in Title Case.}

\begin{longtable}[]{@{}
  >{\raggedright\arraybackslash}p{(\linewidth - 2\tabcolsep) * \real{0.2857}}
  >{\raggedright\arraybackslash}p{(\linewidth - 2\tabcolsep) * \real{0.7143}}@{}}
\toprule\noalign{}
\begin{minipage}[b]{\linewidth}\raggedright
Term / Acronym
\end{minipage} & \begin{minipage}[b]{\linewidth}\raggedright
Definition (EU standard reference)
\end{minipage} \\
\midrule\noalign{}
\endhead
\bottomrule\noalign{}
\endlastfoot
\textbf{AES} -- Advanced Encryption Standard & Symmetric block cipher
standardised in ISO/IEC~18033‑3. \\
\textbf{Algorithm Agility} & Ability of a system to support, select and
switch between multiple cryptographic algorithms with minimal impact
(ETSI~TR~103~684). \\
\textbf{CBOM} -- Cryptography Bill of Materials & Machine‑readable
inventory of algorithms, keys, certificates and crypto modules contained
in a product; extension to CycloneDX v1.6 (CEN/CENELEC draft
prEN~17720). \\
\textbf{CEN} -- Comité Européen de Normalisation & European Committee
for Standardization responsible for non‑electrotechnical standards. \\
\textbf{CENELEC} -- Comité Européen de Normalisation Électrotechnique &
European Committee for Electrotechnical Standardization. \\
\textbf{CRQC} -- Cryptographically Relevant Quantum Computer & Quantum
computer capable of performing Shor‑style attacks on RSA/ECC keys of
practical length (EN~ISO/IEC~2382‑37 draft). \\
\textbf{CRA} -- Cyber Resilience Act & EU regulation proposal on
cyber‑secured products (COM/2022/454). \\
\textbf{CVSS} -- Common Vulnerability Scoring System & Industry standard
for rating IT vulnerabilities (ISO/IEC~30111). \\
\textbf{Dilithium} & Lattice‑based digital‑signature scheme selected by
NIST for standardisation (FIPS~204). \\
\textbf{DORA} -- Digital Operational Resilience Act & EU regulation
2022/2554 on ICT risk management for the financial sector. \\
\textbf{DSSE} -- Delegated Supply‑chain Signing Envelope & JSON envelope
format binding artefact digests and signature metadata (IETF draft). \\
\textbf{ENISA} -- European Union Agency for Cybersecurity & EU agency
providing guidance on cybersecurity and cryptography. \\
\textbf{eIDAS~2} & Regulation (EU)~2024/126 on digital identity and
trust services. \\
\textbf{ETSI} -- European Telecommunications Standards Institute &
Standards body producing ICT technical specs (e.g., ETSI~TS~119~996 on
algorithm agility). \\
\textbf{FAL} -- Forbidden Algorithm List & Organisational list banning
weak or deprecated algorithms (internal policy; reference CRA
Art~23). \\
\textbf{FIPS} -- Federal Information Processing Standard & U.S.
Government cryptography standards (e.g., FIPS~203~ML‑KEM). \\
\textbf{HNDL} -- Harvest‑Now‑Decrypt‑Later & Attack model where
adversary stores encrypted data today to decrypt after CRQC becomes
available (CEN/CENELEC use case). \\
\textbf{HSM} -- Hardware Security Module & Physical device safeguarding
cryptographic keys and operations (ISO/IEC~19790). \\
\textbf{Hybrid Key Exchange} & Protocol combining classical and
post‑quantum Key Encapsulation Mechanisms (KEMs) to derive a shared
secret (IETF~RFC~9399). \\
\textbf{ISO} -- International Organization for Standardization & Global
standardisation body collaborating with IEC on IT. \\
\textbf{Key Governance} & Processes and controls ensuring lifecycle
management of cryptographic keys (EN~ISO/IEC~27002:2022, 10.10). \\
\textbf{Kyber / ML‑KEM} & Module‑lattice KEM selected by NIST
(FIPS~203). \\
\textbf{MTTR‑C} -- Mean Time to Remediate Crypto & Average time to
replace or fix weak cryptography after detection (DORA ICT RTS
draft). \\
\textbf{NIS~2} & Directive (EU)~2022/2555 on measures for a high common
level of cybersecurity across the Union. \\
\textbf{OCSP} -- Online Certificate Status Protocol & Internet X.509
revocation protocol (IETF~RFC~6960). \\
\textbf{PAREK} & Five‑stage EU PQC transition framework:
\textbf{P}‑Inventory, \textbf{A}‑Risk Assessment,
\textbf{R}‑Road‑mapping, \textbf{E}‑Execution, \textbf{K}‑Governance. \\
\textbf{PQC} -- Post‑Quantum Cryptography & Cryptographic primitives
believed secure against quantum adversaries (ISO/IEC~2382‑37 draft
term). \\
\textbf{QACKER} -- Quantum Hacker & Community-driven portal tracking
quantum exploits and proof‑of‑concept attacks on classical cryptography
(\url{https://www.qacker.com}). \\
\textbf{QARS} -- Quantum‑Adjusted Risk Score & Composite metric
weighting shelf‑life, migration effort and threat horizon. \\
\textbf{QTSP} -- Qualified Trust Service Provider & Entity providing
qualified trust services under eIDAS~2. \\
\textbf{RSA} & Public‑key cryptosystem based on integer factorisation
(ISO/IEC~14888‑2). \\
\textbf{SBOM} -- Software Bill of Materials & List of software
components in a product (ISO/IEC~5962:2021 -- SPDX). \\
\textbf{SCB} -- Supplier Cryptography Board & Governance forum reviewing
supplier PQC readiness (§13). \\
\textbf{Shor's Algorithm} & Quantum algorithm for factoring and discrete
logarithms (ISO/IEC~2382‑37 ref). \\
\textbf{SPHINCS+} & Stateless hash‑based signature scheme selected by
NIST (FIPS~205). \\
\textbf{TLS~1.3} & Transport Layer Security protocol version 1.3
(IETF~RFC~8446). \\
\textbf{X.509 Certificate} & Public‑key certificate standard
(ITU‑T~X.509; also ISO/IEC\,9594‑8). \\
\end{longtable}

\subsection{Notes on usage}\label{notes-on-usage}

\begin{itemize}
\tightlist
\item
  Where CEN/CENELEC or ISO vocabulary provides an exact wording, that
  phrasing is preferred verbatim.
\item
  Internal policy acronyms (e.g., QARS, SCB) are capitalised to signal
  organisational scope.
\item
  Terms introduced by NIST but not yet in ISO (e.g., ML‑KEM) keep NIST
  naming with cross‑reference to pending ISO work items.
\end{itemize}

\section{18 Templates, Check‑lists \& Sample
Artefacts}\label{templates-checklists-sample-artefacts}

\begin{quote}
\textbf{Purpose}~-- catalogue the ready‑to‑use artefacts that accelerate
PAREK implementation: spreadsheets, questionnaires, run‑books and
document stubs. All templates live under the repository's
\textbf{\texttt{assets/templates/}} folder so teams can clone or
download them directly.
\end{quote}

\emph{The table lists each template, its intended use, recommended
format and repository path.}

\begin{longtable}[]{@{}
  >{\raggedright\arraybackslash}p{(\linewidth - 8\tabcolsep) * \real{0.0652}}
  >{\raggedright\arraybackslash}p{(\linewidth - 8\tabcolsep) * \real{0.3261}}
  >{\raggedright\arraybackslash}p{(\linewidth - 8\tabcolsep) * \real{0.1957}}
  >{\raggedright\arraybackslash}p{(\linewidth - 8\tabcolsep) * \real{0.1739}}
  >{\raggedright\arraybackslash}p{(\linewidth - 8\tabcolsep) * \real{0.2391}}@{}}
\toprule\noalign{}
\begin{minipage}[b]{\linewidth}\raggedright
\#
\end{minipage} & \begin{minipage}[b]{\linewidth}\raggedright
Template name
\end{minipage} & \begin{minipage}[b]{\linewidth}\raggedright
Purpose
\end{minipage} & \begin{minipage}[b]{\linewidth}\raggedright
Format
\end{minipage} & \begin{minipage}[b]{\linewidth}\raggedright
Repo path
\end{minipage} \\
\midrule\noalign{}
\endhead
\bottomrule\noalign{}
\endlastfoot
1 & \textbf{CBOM JSON Schema} & Validate supplier cryptography bills of
materials against CycloneDX extension & \texttt{.json} &
\texttt{assets/templates/cbom-schema/pqcbom-1.6.json} \\
2 & \textbf{Asset Inventory Spreadsheet} & Manual fallback sheet for
systems where automated scanning is not feasible & \texttt{.xlsx} &
\texttt{assets/templates/inventory/inventory‑baseline.xlsx} \\
3 & \textbf{Risk Calculator Notebook} & Jupyter notebook implementing
QARS formula with sample data & \texttt{.ipynb} &
\texttt{assets/templates/risk/qars\_calc.ipynb} \\
4 & \textbf{Supplier Questionnaire (Q‑PQC‑001)} & Collect vendor crypto
posture \& roadmap (Tier~1/2) & \texttt{.docx} &
\texttt{assets/templates/supplier/q‑pqc‑001.docx} \\
5 & \textbf{PQC Contract Annex} & Standard legal clause bundle
(CRA‑ready) & \texttt{.docx} &
\texttt{assets/templates/contracts/pqc‑annex.docx} \\
6 & \textbf{Migration Run‑book Stub} & Markdown skeleton for Stage~E
deployments & \texttt{.md} &
\texttt{assets/templates/execution/migration\_runbook.md} \\
7 & \textbf{Rollback Playbook} & Script + checklist for emergency cipher
rollback & \texttt{.sh} + \texttt{.md} &
\texttt{assets/templates/execution/rollback/} \\
8 & \textbf{KPI Dashboard JSON} & Grafana import for executive KPI panel
& \texttt{.json} & \texttt{assets/templates/kpi/kpi\_dashboard.json} \\
9 & \textbf{Incident Report Form (ENISA style)} & 72‑hour notification
template for NIS~2 major incidents & \texttt{.docx} &
\texttt{assets/templates/incidents/nis2\_incident\_form.docx} \\
10 & \textbf{Lessons‑Learned Retrospective Deck} & Slide deck for
post‑migration review meetings & \texttt{.pptx} &
\texttt{assets/templates/lessons/retro\_template.pptx} \\
\end{longtable}

\subsubsection{How to use}\label{how-to-use}

\begin{enumerate}
\def\labelenumi{\arabic{enumi}.}
\tightlist
\item
  \textbf{Download or clone} the required file from the path above.
\item
  \textbf{Fill in the yellow‑highlighted fields} -- those are mandatory
  for audit.
\item
  \textbf{Version‑control} completed artefacts in your project folder
  (\texttt{/project/\textless{}work‑package\textgreater{}/docs/}).
\item
  \textbf{Submit} via pull request or the SharePoint drop‑off library as
  instructed in §10 or §12.
\end{enumerate}

\subsubsection{\texorpdfstring{Planned additions
\emph{(placeholder)}}{Planned additions (placeholder)}}\label{planned-additions-placeholder}

\begin{longtable}[]{@{}lll@{}}
\toprule\noalign{}
Template & ETA & Owner \\
\midrule\noalign{}
\endhead
\bottomrule\noalign{}
\endlastfoot
KPI auto‑emailer script & Q1~2026 & CGO‑DevOps \\
CBOM→Neo4j import Lambda & Q2~2026 & DevOps‑Infra \\
\end{longtable}

\section{19 Appendices -- Supporting Artefacts \& Deep‑Dive
Material}\label{appendices-supporting-artefacts-deepdive-material}

\begin{quote}
\textbf{Purpose} -- enumerate and briefly describe the supplementary
artefacts that provide extra depth, raw data or worked examples
referenced throughout the handbook. Each appendix lives either as a
standalone Markdown/PDF in \texttt{assets/appendix/} or as an embedded
section below.
\end{quote}

\subsection{Suggested Appendix
Catalogue}\label{suggested-appendix-catalogue}

\begin{longtable}[]{@{}
  >{\raggedright\arraybackslash}p{(\linewidth - 6\tabcolsep) * \real{0.0385}}
  >{\raggedright\arraybackslash}p{(\linewidth - 6\tabcolsep) * \real{0.5096}}
  >{\raggedright\arraybackslash}p{(\linewidth - 6\tabcolsep) * \real{0.2692}}
  >{\raggedright\arraybackslash}p{(\linewidth - 6\tabcolsep) * \real{0.1827}}@{}}
\toprule\noalign{}
\begin{minipage}[b]{\linewidth}\raggedright
ID
\end{minipage} & \begin{minipage}[b]{\linewidth}\raggedright
Working Title
\end{minipage} & \begin{minipage}[b]{\linewidth}\raggedright
Intended Content (summary)
\end{minipage} & \begin{minipage}[b]{\linewidth}\raggedright
Format \& Location
\end{minipage} \\
\midrule\noalign{}
\endhead
\bottomrule\noalign{}
\endlastfoot
\textbf{A} & Algorithm \& Parameter Cheat‑Sheets & One‑page tables for
ML‑KEM, ML‑DSA, SPHINCS+ parameters (security level, key sizes,
cipher‑suite IDs). Useful for architects and auditors. &
\texttt{assets/appendix/A-algo-cheatsheet.pdf} \\
\textbf{B} & CRQC Scenario Planning Worksheets & Excel model pre‑loaded
with Fast/Baseline/Delayed threat horizons (§9.7) and Monte Carlo
templates for budget impacts. &
\texttt{assets/appendix/B-scenario-worksheets.xlsx} \\
\textbf{C} & EU Regulatory Mapping Matrix & Pivot table mapping handbook
sections, KPIs and reference architectures to NIS~2, DORA, CRA, eIDAS~2
articles. & \texttt{assets/appendix/C-eu-reg-matrix.xlsx} \\
\textbf{D} & Sample CBOM \& Validation Report & Example Tier‑1 supplier
CBOM JSON, DSSE signature, and automated validation output. &
\texttt{assets/appendix/D-sample-cbom/} \\
\textbf{E} & Q-PQC-001 Supplier Questionnaire & Blank and filled‑in
versions demonstrating expected depth of answers. &
\texttt{assets/appendix/E-supplier-questionnaire.docx} \\
\textbf{F} & KPI Dictionary & Extended definitions, formulas and
SQL/Grafana queries for every KPI in §15. &
\texttt{assets/appendix/F-kpi-dictionary.md} \\
\textbf{G} & Risk Formula Reference & Derivation of QARS weights,
academic citations, sensitivity analysis plots. &
\texttt{assets/appendix/G-qars-formula.pdf} \\
\textbf{H} & Contract Annex Boilerplate & Full text of the PQC Annex
with tracked‑changes commentary and CRA cross‑references. &
\texttt{assets/appendix/H-pqc-annex.docx} \\
\textbf{I} & Training Matrix \& Syllabi & Detailed curricula, slide
decks and lab guides for Dev‑Ops, IT~Ops and Risk roles. &
\texttt{assets/appendix/I-training-matrix/} \\
\textbf{J} & Tool Installation Scripts & Bash/PowerShell scripts to
deploy reference stack (OQS‑OpenSSL, Envoy, Neo4j). &
\texttt{assets/appendix/J-tool-scripts/} \\
\end{longtable}

Feel free to re‑letter or reorder appendices as the handbook matures;
maintain unique IDs for citation stability.

\subsection{Next‑step actions}\label{nextstep-actions}

\begin{enumerate}
\def\labelenumi{\arabic{enumi}.}
\tightlist
\item
  \textbf{Content owners} -- populate each appendix folder/file before
  handbook v1.0 freeze (target Q1~2026).
\item
  \textbf{Editorial review} -- ensure consistency with glossary and
  policy stack.
\item
  \textbf{Linking} -- update in‑text references (e.g., ``see
  Appendix~C'') as each appendix finalises.
\end{enumerate}

% Footer note or appendix
\newpage
\thispagestyle{empty}
\vfill
\begin{center}
  \textit{This document was generated using the PAREK open framework.}
\end{center}

\end{document}
